\documentclass[a4paper,12pt]{article}
%\usepackage[a4paper, portrait, margin=0.5in]{geometry}
\usepackage[utf8]{inputenc}
\usepackage[spanish]{babel}
\selectlanguage{spanish}
\usepackage{amsmath}
\usepackage{amsfonts}
\usepackage{polynomial}
\usepackage{makeidx}
\usepackage{graphicx}
\usepackage{lmodern}

\begin{document}
\newcommand{\osf}[2]{\dfrac{#1^2}{#2^2}}
\newcommand{\T}{\Big(2\pi\cdot\sqrt{\dfrac{l}{g}}\Big)}
\newcommand{\alpa}{\dfrac{e^{x \beta -3}{2y\beta +5}}}
\newcommand{\deriv}[2]{\dfrac{\delta #1}{\delta #2}}
\newcommand{\ext}{e^{ax-bt}}
\newcommand{\exy}{e^{((x-1)^2 + (y-3)^2}}
\providecommand{\norm}[1]{\parallel #1\parallel}
\begin{center}
	\textbf{Matemáticas para las ciencias aplicadas III, Tarea I}\\
	\textbf{\textit{Arteaga Vázquez Alan Ernesto, Almeida Rodríguez Jerónimo y Sánchez Salgado Alma Rocío}}
\end{center}

\section{\texttt{Mardsen - Tromba | Sección 3.3}}
\textit{\textbf{$27.$}Suppose $f: \mathbb{R}^3 \mapsto \mathbb{R}$ is $C^2$, and that $x_{0}$ is a critical point for $f$. Suppose $Hf (x_{0})(h) = h_{1}^2 + h_{2}^2 + h_{3}^2 + 4h_{2}h_{3}$ Does $f$ have a local maximum, minimum, or saddle a $x_{0}$?}\\

Tenemos que $Hf(x_0)(h) = h_1^2+h_2^2+h_3^2++4h_2  h_3$, si factorizamos, entonces: 
\[Hf(x_0)(h)= h_1^2+(h_2+h_3)^2+2h_2h_3\]
\[g \left( \dfrac{h}{ \parallel h \parallel}\right). \parallel h \parallel^2\]
donde $\norm{h} ^2 = h_1^2+h_2^2+h_3^2$ y $g $ una función. $H(h)\geq 1. \norm{h}^2$ entonces, dependiendo de $h_2, h_3$  $H(h)$ puede ser positivo o negativo y así tener un local mínimo o máximo o un punto de ensilladura.
Sea $h_1=h_2=h_3$, luego $H(h)= 7h^2$ siendo positivo (mínimo local)
$h_1=h_3= h$, $h_2 = -h$, luego $H(h) = -h^2$ siendo negativo (máximo local). 

Por lo tanto, el punto críto es un punto de ensilladura, en una dirección tiene mínimo y en otra tiene un máximo.


\textit{\textbf{51.}}
	\begin{itemize}
		\item[(a)] Let f be a $C^1$ function on the real line $\mathbb{R}$. Suppose that f has exactly one critical point $x_{0}$ that is a strict local minimum for $f$ . Show that $x_{0}$ is also an absolute minimum for $f$; that is, that $f(x) \geq f(x_{0})$ for all $x$.
		
		Como $f$ tiene exactamente un punto crítico en $x_{0}$ y es una función de $C'$, entonces : 
		$f'(x) = 0$ para $x= x_{0}$ y $f'(x)\neq 0$ para $x\neq x_{0}$
		$f$ es continua en la recta real, entonces, ya sea que $f$ aumente o decremente en cualquier punto $x\neq x_{0}$ . Como $x_{0}$ es un mínimo local, tenemos que : 
		$f(x_0)< f(x)$ para todo $0<| x-x_{0}| < \delta$, entonces para $x> x_{0}$ la derivada de $x$ será mayor a cero y para $x<x_{0}$ la derivada de $f$ evaluada en $x$ será menor a cero. 
		Luego, $f$ está decreciendo para toda $x$ tal que $x<x_{0}$, así $f(x)\geq f(x_0)$ para toda $x$ que pertenece a los reales y $x_0$ es también un mínimo absoluto de $f$
		\item[(b)] The next example shows that the conclusion of part (a) does not hold for functions of more tha one variable. Let $f: \mathbb{R}^2 \mapsto \mathbb{R}$ be defined by 
		\[f(x,y) = -y^4-e^{-x^2}+2y^2 \sqrt{e^x + e^{-x^2}}\] 
		show that $(0,0)$ is the only critical point for $f$ and that is a local minimum
		\[\bigtriangledown f = \left[ 2xe^{-x^2}+2y^2 \dfrac{\left( e^x -2xe^{-x^2} \right)}{2\sqrt{e^x+e^{-x^2}}}\right]i +\left[-4y^3+4y \sqrt{e^x+e^{-x^2}} \right]j\]
		Ahora $\dfrac{\delta f}{\delta x} = 0$, entonces : 
		\[2xe^{-x^2}+ \dfrac{2y^2 \left(e^x-ex e^{-x^2} \right)}{2\sqrt{e^x + e^{-x^2}}}= 0\]
		\[y^2(e^x-2x e^{-x^2})= -2xe^{-x^2}\sqrt{e^x+ e^{-x^2}}\]
		\[y^4 \left(e^{2x}+4x^2 e^{x-x^2} \right) = 4x^2e^{-2x^2}\left(e^x + e^{-x^2}\right)\]
	$(1):$
		\[y^4\left(e^{2x}+4x^2e^{-2x^2}-4xe^{x-x^2} \right)+ 4x^2e^{-2x^2}\left( e^x+ e^{-x^2}\right)=0 \]
		También $\dfrac{\delta f}{\delta y} = 0$, entonces $4y\left(-y^2+ \sqrt{e^x + e^{-x^2}}\right)=0$. Así $y=0$ o $(2):$ 
		\[y^2= \sqrt{e^x+ e^{-x^2}}\]
		Cuando $y = 0$ de $(1)$
		\[2x \left(e^{-x^2+x}+ e^{-2x^2} \right)=0\]
		Entonces $x=0$, así que un punto crítico es $(0,0)$
		Ahora, cuando $y = \sqrt{e^x+ e^{-x^2}}$, de $(1)$ tenemos que :
		\[(e^x+ e^{-x^2})(e^{2x}+4x^2e^{-2x^2}-4xe^{x-x^2})+2xe^{-x^2}(e^x+ e^{-x^2})=0\]
		Ó : 
		\[(e^x+e^{-x^2})x(e^x-2xe^{-x^2})^2 +2xe^{-x^2}=0\]
		Ó :
		\[2x=\dfrac{-\left( e^x- 2xe^{-x^2}\right)^2}{2e^{-x^2}}\]
		y esta última no cumple para algún $x\epsilon R$, así que el único punto crítico es $(0,0)$
		Para ver que es un mínimo local : 
		\[\dfrac{\delta^2 f}{\delta x^2}=\dfrac{-4x^2e^{-x^2}+2e^{-x^2}+y^2 \left[ \left( e^x+ 4x^2-2e^{-x^2}\right)x \sqrt{e^x+e^{-x^2}}- \dfrac{\left(e^x- 2xe^{-x^2} \right)^2}{2 \sqrt{e^x+ e^{-x^2}}} \right]}{\left(e^x+ e^{-x^2} \right)}\]
		entonces : $\dfrac{\delta ^2 f}{\delta y^2}(0,0) = 2 $ y también $\dfrac{\delta ^2 f}{\delta y^2}= -12y^2+ 8 \sqrt{e^x + e^-x^2}$
		y $\dfrac{\delta ^2 f}{\delta y^2}(0,0)= 8 \sqrt{2}$ Y $\dfrac{\delta^2 f}{\delta y \delta x}= 2y \dfrac{(e^x- 2xe^{-x^2})}{\sqrt{e^x+ e^{-x^2}}}$, también $\dfrac{\delta^2 f(0,0)}{\delta y \delta x}= 0$, ahora :
		\[\left(\dfrac{\delta^2 f}{\delta y \delta x}(0,0) \right)^2 - \dfrac{\delta^2 f}{\delta x^2}(0,0).\dfrac{\delta ^2 f}{\delta y^2}(0,0) = -16\sqrt{2}<0\]
		También : 
		\[\dfrac{\delta ^2 f}{\delta x^2}(0,0)+\dfrac{\delta ^2 f}{\delta y^2}(0,0)= 2+8\sqrt{2}> 0\]
		Por lo tanto, $(0,0)$ es un punto mínimo local.
	\end{itemize}

\section{\texttt{Mardsen - Tromba | Sección 3.4}}

\textit{\textbf{26.}An irrigation canal in Arizona has concrete sides and bottom with trapezoidal cross section of area
 $A = y(x + y \tan \theta)$ and wetted perimeter \\
 $P = x + \dfrac{2y}{\cos \theta}$, where $x = $bottom width, $y = $water depth, and $\theta =$side inclination, measured
 from vertical.The best design for a fixed inclination $\theta$ is found by solving $P =$ minimum subject to the condition
 $A =$ constant. Show that $y^2 =\dfrac{A \cos \theta}{2 - \sin \theta}$ 
}

Tenemos que $f(x,y, \lambda) = x+ 2y \sec \theta + \lambda(xy+ y^2 tan\theta - A)$
Ahora $\dfrac{\delta f}{\delta x}= 1+ \lambda y = 0$, entonces $\lambda = \dfrac{-1}{y}$
\[\dfrac{\delta f}{\delta y} = 2 \sec \theta + \lambda(x + 2y\tan \theta)= 0\]
\[\lambda = \dfrac{-2 \sec\theta}{x+2y \tan \theta}\] 
y 
\[\dfrac{\delta f}{\delta \lambda} = xy+ y^2 \tan \theta - A =0 \Longleftrightarrow A = xy+ y^2 \tan \theta\]
Igualamos los valores de $\lambda$, entonces tenemos :
\[2y = \dfrac{x+ 2y \tan \theta}{\sec \theta}= \dfrac{x\cos \theta+ 2y \sin \theta}{1}\]
\[\Rightarrow 2y(1-\sin \theta) = x \cos \theta\]
\[\Rightarrow x = \dfrac{2y(1-\sin \theta)}{\cos \theta}= 2y (\sec \theta - \tan \theta)\]
Susstituyendo dentro $\dfrac{\delta f}{\delta \lambda} = 0$ tenemos que :
\[A = 2y^2(\sec \theta - \tan \theta)+ y^2\tan \theta = y^2 (2 \sec\theta - \tan \theta)\]
\[\Rightarrow y^2 = \dfrac{A}{2\sec \theta - \tan \theta}\]
\[y^2 = \dfrac{A \cos \theta}{2- \sin \theta}\]
\section{\texttt{Hugues - Hallet | Sección 16.1}}
\textit{\textbf{5.} Figure $16.8$ shows a contour plot of population density, people per square kilometer, in a rectangle of land $3$ km by $2$ km. Estimate the population in the refion represented by Figure $16.8$}

Sea R el rectángulo y $f(x,y)$ la densidad poblacional en el punto $(x,y)$, la población está dada por $\int_{R}f(x,y)dA$.. Si aproximamos la integral con sumas de Riemann usando la información de cada km por km, entonces podemos tomar la densidad en un un punto y sumarlos. 
Vamos a dividir el eje $y$ en dos partes y el eje $x$ en tres, tenemos los puntos: 
(0.5, 1.5)=500, (0.5, 0.5)= 650, (1.5,1.5)= 450, (1.5, 0.5)=200, (2.5, 1.5)=100 y (2.5, 0.5)= 400
por lo que la densidad por km cuadrado es la suma de estos valores
\[\int_{R}(x,y)dA\approx 500 + 450 + 100 + 650 + 200 + 400 = 2300\]

\textit{\textbf{6-12} Decide(without calculation) wheter the integrals are positive, negative, or zero. Let $D$ be the region inside the unit circle centered at the origin, let $R$ be the rigth half of $D$ and let $B$ be the bottom half of $D$}
\begin{itemize}
	\item[6.]$\int_{D}dA$
	         Es positiva, la integral de $f(x,y) = 1$
	\item[7.]$\int_{R}5xdA$
			  La integral es positiva, tomemos en cuenta que es la mitad derecha del área total del círculo unitario, entonces toma x toma valores mayores a $0$. La función siendo integrada es $f(x,y) = 5x$
	\item[8.]$\int_{B}5xdA$
			  B es una función simétrica respecto a $x$ por lo que la integral se cancela: $f(x,y) = -f (-x,y)$ y el resutado es cero.
	\item[9.]$\int_{D}(y^3 + y^5)dA$
		  	   D es simétrico respecto a $y$, por lo que la integral es cero también
	\item[10.]$\int_{B}(y^3+y^5)dA$
			   Los valores de la función con $y < 0$ (porque es la mitad de abajo del círculo)son negativos,por lo que la integral también lo es.
	\item[11.]$\int_{D}(y-y^3)dA$
				$f(x,y) = y- y^3$ es una función impar, y D es simétrica respecto a $y$ por lo que la integral se cancela y es igual a cero
	\item[12.]$\int_{B}(y-y^3)dA$	
			 La función $f(x,y) = y -y^3$ es negativa en la región B ya que todos los valores de $y$ en esa región son negativos, entonces tenemos que $\mid y^3 \mid < \mid y \mid$ por lo que la integral es negativa. 
\end{itemize}

\section{\texttt{Anton - Bivens - Davis | Sección 14.1}}
\textit{\textbf{10.} $\int^{\pi}_{\pi / 2}\int^2_1 x \cos xy dy dx$}\\

Primero resolvemos la integral $\int^2_1 x \cos xy dy$ con cambio de variable. Sea $u= xy$, entonces si derivamos con respecto a $y$,  $du = x dy$, por lo que la integral ahora es 
\[\int^2_1 \cos u du = \sin u\mid ^2 _1\]
\[= \sin(xy) \mid^{y=2}_{y=1}\]
Y ahora resolvemos la doble integral sustituyendo : 
\[\int^\pi_{\pi / 2} \sin(xy) \mid^{y=2}_{y=1} dx\]
\[\int^\pi_{pi / 2} \sin(2x)-\sin(x) dx\]
Así: 
\[\int^\pi_{pi / 2} \sin(2x) dx - \int^\pi_{pi / 2} \sin(x) dx\]
Para la primera integral, definimos a $u$ como $u = 2x$ entonces $du = 2 dx$, luego $dx = 1/2 du$
\[1/2 \int^\pi_{pi / 2}\sin(u) du - \int^\pi_{pi / 2} \sin(x) dx\]
\[1/2(-\cos(2x)\mid^\pi_{\pi / 2})-(-\cos(x)\mid^\pi_{\pi / 2}) \]
\[= 1/2[-\cos(2\pi)+cos(\pi)]+ [\cos(\pi)-\cos(\pi / 2)] \]
Evaluamos : 
\[= 1/2(-1-1)+(-1+0) = -2\]
Entonces: 
\[\int^{\pi}_{\pi / 2}\int^2_1 x \cos(xy) dy dx = -2\]
\textit{\textbf{28.}Use the result in Exercise 27 to evaluate the integral 
\[\int^{ln 3}_{0} \int_1 ^1 \sqrt{e^y + 1}\tan x dx dy\] by inspection. Explain your reasoning}\\

Del ejercicio $27$ tenemos que:
\[\int_{R}\int f(x,y)dA = \int_a^b[\int_c^dg(x)h(y)dy]dx\]
\[\int_a^b g(x)[\int_c^dh(y)dy ]dx\]
\[[\int_a^b g(x) dx][\int_c^d h(y) dy]\]
Entonces, sustituyendo tenemos que 
\[\int^{ln 3}_{0} \int_1 ^1 \sqrt{e^y + 1}\tan x dx dy=[\int_o^{ln3} \sqrt{e^y + 1}dy][\int_1^1tan x dx]\]
Sabemos que $tan x$ es una función impar, por lo que al ser evaluada con $1$ y $-1$ es igual a cero, por lo tanto la integral es $0$\

\textit{\textbf{33.}Evaluate the integral by choosing a convenient order of integration:}
\[\int \int_{R}xcos(xy)\cos^2 \pi x dA; R = [0, \dfrac{1}{2}] x [0, \pi] \]

Tenemos la integral:
 \[\int_{0}^{1/2}\int_{0}^\pi x\cos(xy) \cos^2 (\pi x) dy dx\]
El orden por el que optaremos será el siguiente: 
\[\int_{0}^{1/2} \left[ \int_{0}^{\pi} x\cos(xy) \cos^2(\pi x)dy \right] dx \]
Resolvemos primero:
\[\int_{0}^{\pi} x\cos(xy) \cos^2(\pi x)dy\]
Como estamos integrando respecto a $y$, entonces podemos sacar las constantes, así:
\[x\cos^2(\pi x)\int_{0}^{\pi} \cos(xy) dy\]
Hacemos cambio de variable, sea $u = xy$, entonces $du = x dy$ y $dy = du \left( \dfrac{1}{x} \right)$
\[x\cos^2(\pi x)\int_{0}^{\pi} \cos(xy) dy = x\cos^2(\pi x)\left[ \dfrac{1}{x} \int_{0}^{\pi}\cos u du \right] \]
\[= x\cos^2(\pi x)\left[\dfrac{1}{x}\left(\sin(xy)|_{o}^{\pi}\right) \right]\]
\[= x\cos^2(\pi x)\left[\dfrac{1}{x} \left(\sin(\pi x)- \sin(0) \right) \right]= x\cos^2(\pi x)\left[ \dfrac{1}{x} \left(\sin(\pi x)\right) \right]\]
\[= \cos^2(\pi x) \sin(\pi x) \left(\dfrac{1}{x} \right)= \cos^2(\pi x)\sin(\pi x)\]
Entonces : 
\[\int_{0}^{1/2} \left[ \int_{0}^{\pi} x\cos(xy) \cos^2(\pi x)dy \right] dx = \int_{0}^{1/2}\cos^2(\pi x)\sin(\pi x) dx\]
\[=\int_{0}^{1/2}\sin(\pi x)\cos^2(\pi x) dx\]
Sea $u= \pi x$, entonces $du= \pi dx$ y $dx= \dfrac{1}{\pi}du$, así: 
\[=\dfrac{1}{\pi} \int_{0}^{\pi/2}\sin u \cos^2u du\]
Hagamos un cambio de variable más, sea $v = \cos u$, entonces $dv = -\sin(u)du$ por lo que:
\[= \dfrac{1}{\pi} \int_{0}^{\pi / 2}v^2 dv = -\dfrac{1}{\pi}\left[\dfrac{v^3}{3}|_{0}^{\pi / 2}\right]\]
\[=-\dfrac{1}{\pi}\left[\dfrac{\cos(\pi x)}{3} |_{0}^{1/2}\right]= -\dfrac{1}{\pi}\left[\dfrac{\cos(\pi x)}{3}|_{0}^{1/2}\right]\]
\[-\dfrac{1}{\pi}\left[\dfrac{1}{3} \cos(\pi / 2)- \cos(0)\right]= \dfrac{1}{3\pi}\]
\end{document}