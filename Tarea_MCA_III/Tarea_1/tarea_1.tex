\documentclass[a4paper,12pt]{article}
%\usepackage[a4paper, portrait, margin=0.5in]{geometry}
\usepackage[utf8]{inputenc}
\usepackage[spanish]{babel}
\selectlanguage{spanish}
\usepackage{amsmath}
\usepackage{amsfonts}
\usepackage{polynomial}
\usepackage{makeidx}
\usepackage{graphicx}
\usepackage{lmodern}

\begin{document}
\newcommand{\osf}[2]{\dfrac{#1^2}{#2^2}}
\newcommand{\T}{\Big(2\pi\cdot\sqrt{\dfrac{l}{g}}\Big)}
\newcommand{\alpa}{\dfrac{e^{x \beta -3}{2y\beta +5}}}
\newcommand{\deriv}[2]{\dfrac{\delta #1}{\delta #2}}
\newcommand{\ext}{e^{ax-bt}}
\newcommand{\exy}{e^{((x-1)^2 + (y-3)^2}}
\begin{center}
	\textbf{Matemáticas para las ciencias aplicadas III, Tarea I}\\
	\textbf{\textit{Arteaga Vázquez Alan Ernesto, Almeida Rodríguez Jerónimo
					y Sánchez Salgado Alma Rocío}}
\end{center}

\newpage

\section{Marsden - Tromba | Sección 3.3}

\textit{23. An examination of the function}
	$$f:\mathbb{R}^2 \mapsto \mathbb{R}, (x, y) \mapsto (y -3x^2)(y -x^2)$$
	\textit{will give an idea of the difficulty of finding conditions that
			guarantee that a critical point is a relative extremum when Theorem
			6 fails. Show that}\\

	\textit{a) the origin is a critical point of $f$ ;}\\

	Procedemos a derivar parcialmente la función, así, se sigue:
	$$f(x,y) = y^2 -3x^2y -x^2y + 3x^4 = y^2 -4x^2y + 3x^4  $$
	$$\frac{\partial f}{\partial x} = -8xy + 12x^3 $$
	$$\frac{\partial f}{\partial y} = 2y - 4x^2 $$

	luego, para ver que precisamente el origen es un punto crítico,
	evaluamos las parciales en dicho punto, así:
	$$\left. \frac{\partial f}{\partial x} \right|_{(0,0)}
			= -8(0)(0) + 12(0)^3 = 0 $$
	$$\left. \frac{\partial f}{\partial y} \right|_{(0,0)}
			= 2(0) - 4(0)^2 = 0 $$

	así, como todas las parciales son cero en el origen, se sigue que este es
	un punto crítico.\\

	\textit{b) $f$ has a relative minimum at $(0, 0)$ on every straight line
			through $(0, 0)$; that is, if $g(t) = (at, bt)$, then
			$f \circ g : \mathbb{R} \rightarrow \mathbb{R}$ has a relative
			minimum at 0,for every choice of a and b;}\\

		Consideremos ahora a la funcíón $f \circ g$:
		$$ f \circ g = (bt)^2 - 4(at)^2(bt) + 3(at)^4$$
		$$ = b^2t^2 - 4a^2bt^3 + 3a^4t^4$$

		ahora, derivando la función compuesta, se tiene:
		$$\frac{\partial f}{\partial t} = 2b^2t -12a^2bt^2 + 12a^4t^3
										= 2t(b^2 - 6a^2bt + 6a^4t^2)$$

		se sigue que $t = 0$ es un punto crítico correspondiente a $(x,y) = (0,0)$.
		Ahora, derivando nuevamente la función, se tiene:

		$$\frac{\partial^2 f}{\partial t^2} = 2b^2 -24a^2bt + 36a^4t^2 $$

		ahora, evaluando la segunda derivada en el punto $t = 0$
		$$\left. \frac{\partial^2 f}{\partial t^2} \right|_{0}
			= 2b^2 -24a^2b(0) + 36a^4(0)^2 = 2b^2 $$

		como
			$$\left. \frac{\partial^2 f}{\partial t^2} \right|_{0} = 2b^2 > 0$$
		se sigue que $t = 0$ es un mínimo local para
		$\forall a$ si $b \neq 0$\\

		Ahora, si $b = 0$, se cumple que
			$$f \circ g (t) = 3(at)^4$$
		así que $t = 0$ resulta ser un mínimo local para cualquier elección de
		$a$ y $b$.\\

	\textit{(c) the origin is not a relative minimum of $f$ .}\\
	\newline


	\textit{41. Find the absolute maximum and minimum values for
		$$f(x, y) = sin x + cos y$$ on the rectangle $R$ defined by
		$$0 \leq x \leq 2\pi, 0 \leq y \leq 2\pi$$}

		Derivando parcialmente, se tiene:
		$$\frac{\partial f}{\partial x} = \cos x $$
		$$\frac{\partial f}{\partial y} = - \sin x $$

		ahora, para los intervalos $0 \leq x \leq 2\pi, 0 \leq y \leq 2\pi$
		se cumple:
			$$\frac{\partial f}{\partial x} = 0 \quad \text{si} \quad
				x=\frac{\pi}{2}, \frac{3\pi}{2}$$
			$$\frac{\partial f}{\partial y} = 0 \quad \text{si} \quad
				y=0, \pi, 2\pi$$

		así, se sigue que la función tiene puntos críticos en:
			$$(\frac{\pi}{2}, 0), (\frac{\pi}{2}, \pi), (\frac{\pi}{2}, 2\pi)$$
			$$(\frac{3\pi}{2}, 0), (\frac{3\pi}{2}, \pi), (\frac{3\pi}{2}, 2\pi)$$

		ahora, construimos la matriz Hessiana para la función derivando
		parcialmente por segunda vez a la función, así:
			$$ \frac{\partial^2f}{\partial x^2} = -\sin x$$
			$$ \frac{\partial^2f}{\partial x\partial y} = 0 = \frac{\partial^2f}{\partial y\partial x}$$
			$$ \frac{\partial^2f}{\partial y^2} = - \cos y$$

			$$H(x,y) = \begin{bmatrix}
    					\dfrac{\partial^2f}{\partial x^2} & & \dfrac{\partial^2f}{\partial x\partial y} \\
    					& & \\
    					\dfrac{\partial^2f}{\partial y\partial x}&  & \dfrac{\partial^2f}{\partial y^2} \\
						\end{bmatrix} =
						\begin{bmatrix}
			    			-\sin x & 0 \\
			    			0  & - \cos y \\
						\end{bmatrix}
						$$

		luego, evaluando la matriz en los puntos críticos y sacando su determinante,
		se tiene:
		$$det[H(\frac{\pi}{2}, 0)] =
			\begin{vmatrix}
				-\sin \dfrac{\pi}{2} & 0 \\
				0  & - \cos 0 \\
			\end{vmatrix} =
			\begin{vmatrix}
				-1 & 0 \\
				 0 & -1 \\
			\end{vmatrix} = 1$$

		Como $f_{xx} < 0$ y $det(H) > 0$, entonces corresponde a un máximo local.\\
		$$det[H(\frac{\pi}{2}, \pi)] =
			\begin{vmatrix}
				-\sin \dfrac{\pi}{2} & 0 \\
				0  & - \cos \pi \\
			\end{vmatrix} =
			\begin{vmatrix}
				-1 & 0 \\
				 0 & 1 \\
			\end{vmatrix} = -1$$
		Como $f_{xx} < 0$ y $det(H) < 0$, entonces corresponde a un punto de ensilladura.\\
		$$det[H(\frac{\pi}{2}, 2\pi)] =
			\begin{vmatrix}
				-\sin \dfrac{\pi}{2} & 0 \\
				0  & - \cos 2\pi \\
			\end{vmatrix} =
			\begin{vmatrix}
				-1 & 0 \\
				 0 & -1 \\
			\end{vmatrix} = 1$$
		Como $f_{xx} < 0$ y $det(H) > 0$, entonces corresponde a un máximo local.\\
		\newpage


		$$det[H(\frac{3\pi}{2}, 0)] =
			\begin{vmatrix}
				-\sin \dfrac{3\pi}{2} & 0 \\
				0  & - \cos 0 \\
			\end{vmatrix} =
			\begin{vmatrix}
				1 & 0 \\
				 0 & -1 \\
			\end{vmatrix} = -1$$

		Como $f_{xx} > 0$ y $det(H) < 0$, entonces corresponde a un punto de ensilladura.\\
		$$det[H(\frac{3\pi}{2}, \pi)] =
			\begin{vmatrix}
				-\sin \dfrac{3\pi}{2} & 0 \\
				0  & - \cos \pi \\
			\end{vmatrix} =
			\begin{vmatrix}
				1 & 0 \\
				 0 & 1 \\
			\end{vmatrix} = 1$$
		Como $f_{xx} > 0$ y $det(H) > 0$, entonces corresponde a un mínimo local.\\
		$$det[H(\frac{3\pi}{2}, 2\pi)] =
			\begin{vmatrix}
				-\sin \dfrac{3\pi}{2} & 0 \\
				0  & - \cos 2\pi \\
			\end{vmatrix} =
			\begin{vmatrix}
				1 & 0 \\
				 0 & -1 \\
			\end{vmatrix} = -1$$
		Como $f_{xx} > 0$ y $det(H) < 0$, entonces corresponde a un punto de ensilladura.\\


		Luego, evaluando máximos locales en la función para determinar
		máximos absolutos:
			$$ f(\frac{\pi}{2}, 0) = \sen(\frac{\pi}{2}) + cos(\pi) = 1 + 1 = 2 $$
			$$ f(\frac{\pi}{2}, 2\pi) = \sen(\frac{\pi}{2}) + cos(2\pi) = 1 + 1 = 2$$

		así, determinamos que la función $f$ tiene máximos absolutos en los puntos
			$$(\frac{\pi}{2}, 0), (\frac{\pi}{2}, 2\pi)$$
		y mínimo absoluto en
			$$(\frac{3\pi}{2}, \pi)$$

	\section{Marsden - Tromba | Sección 3.4}
	\section{Hughes-Hallet | Sección 16.1}
	\section{Anton-Bivens-Davis | Sección 14.1}

\end{document}
