\documentclass[a4paper,12pt]{article}
\usepackage[a4paper, portrait, margin=0.5in]{geometry}
\usepackage[utf8]{inputenc}
\usepackage[spanish]{babel}
\selectlanguage{spanish}
\usepackage{amsmath}
\usepackage{amsfonts}
\usepackage{polynomial}
\usepackage{makeidx}
\usepackage{graphicx}
\usepackage{lmodern}

\begin{document}
\newcommand{\osf}[2]{\dfrac{#1^2}{#2^2}}
\newcommand{\T}{\Big(2\pi\cdot\sqrt{\dfrac{l}{g}}\Big)}
\newcommand{\alpa}{\dfrac{e^{x \beta -3}{2y\beta +5}}}
\newcommand{\deriv}[2]{\dfrac{\delta #1}{\delta #2}}
\newcommand{\sderiv}[3]{\dfrac{\delta ^2 #1}{\delta #2 \delta #3}}
\newcommand{\ext}{e^{ax-bt}}
\newcommand{\exy}{e^{((x-1)^2 + (y-3)^2}}
\newcommand{\xo}{\vec{x_0}} %Devuelve un vector x_0
\providecommand{\norm}[1]{\parallel #1\parallel}

\begin{titlepage}
	\centering
	{\scshape\LARGE Universidad Autónoma de México \par}
	\vspace{1cm}
	{\scshape\Large Matemáticas para las Ciencias Aplicadas III\par}
	\vspace{1.5cm}
	{\huge\bfseries Tarea I\par}
	\vspace{.5cm}
	{\Large\itshape Alan Ernesto Arteaga Vázquez \par}
    \vspace{.5cm}
	{\Large\itshape Alma Rocío Sánchez Salgado \par}
    \vspace{.5cm}
	{\Large\itshape Jerónimo Almeida Rodríguez \par}
	\vfill
	 \includegraphics[width=0.5\textwidth]{../escudo_f-ciencias.png}
	\vfill

% Bottom of the page
	{\large Jueves 23 de Agosto del 2018 \par}
\end{titlepage}

\newpage
Link
\section{Marsden - Tromba | Sección 3.3}


\textit{17. Find all local extrema for $f (x, y) = 8y^3 +12x^2 - 24xy$.}\\

Procedemos a derivar parcialmente la función para $x, y$ así:

	$$ \frac{\partial f}{\partial x} = 24x - 24y$$
	$$ \frac{\partial f}{\partial y} = 24y^2 - 24x$$

luego, para hallar los extremos locales, igualamos las parciales a cero, así:
	$$ \frac{\partial f}{\partial x} = 24x - 24y = 0$$
	$$ 24x = 24y$$
	$$ x = y$$

y también:
	$$ \frac{\partial f}{\partial y} = 24y^2 - 24x = 0$$
	$$ 24y^2 - 24x = 0$$
	$$ 24y^2 - 24y = 0$$
	$$ 24y(y - 1) = 0$$

de allí se sigue:
	$$ y = 0; \; y=1 $$

con lo que, como $x = y$ y $ y = 0; \; y=1 $, se sigue que los puntos críticos
son:
	$$ (0,0), (1,1) $$

luego, procedemos a derivar iteradamente para poder construir la matriz Hessiana
y así determinar la naturaleza de los puntos críticos, así se sigue:
	$$ \frac{\partial^2 f}{\partial x^2} = 24$$
	$$ \frac{\partial^2 f}{\partial x \partial y } = -24
	 = \frac{\partial^2 f}{\partial y \partial x } $$
	$$ \frac{\partial^2 f}{\partial y^2} = 48y$$

Así, contruyendo la matriz Hessiana y evaluando en los puntos críticos, se sigue:
	$$H(x,y) = \begin{bmatrix}
			\dfrac{\partial^2f}{\partial x^2} & & \dfrac{\partial^2f}{\partial x\partial y} \\
			& & \\
			\dfrac{\partial^2f}{\partial y\partial x}&  & \dfrac{\partial^2f}{\partial y^2} \\
			\end{bmatrix} =
			\begin{bmatrix}
				24 & -24 \\
				-24  & 48y \\
			\end{bmatrix}
			$$
	$$det [H (0,0)] =
			\begin{vmatrix}
				24 & -24 \\
				-24  & 0 \\
			\end{vmatrix} = -(-24)(-24) = -576 < 0
	$$
	Como $f_{xx} > 0$ y $det(H) < 0$, entonces $(0,0)$ corresponde a un
	punto de ensilladura (el cual no constituye un extremo local).\\

	Análogamente:\\
	$$det [H (1,1)] =
			\begin{vmatrix}
				24 & -24 \\
				-24  & 48 \\
			\end{vmatrix} = (24)(48) -(-24)(-24) = 1152 -576 = 576 > 0
	$$
	Como $f_{xx} > 0$ y $det(H) > 0$, entonces $(1,1)$
	corresponde a un mínimo local, siendo este un punto extremo.\\


\textit{18. Sea $f(x,y,z) = x^2 + y^2 + z^2 + kyz$}
\begin{itemize}
	\item[a] Comprobar que $(0,0,0)$ es punto crítico para $f$.
	$$\nabla f = \langle \deriv{f}{x}, \deriv{f}{y}, \deriv{f}{z}\rangle = \langle 2x, 2y + kz, 2z +ky \rangle$$
	Igualando las parciales a $0$, tenemos que
	\begin{itemize}
		\item[i] Si $2x = 0$, entonces $x = 0$.
	s	\item[ii] Si $2y + kz = 0$, entonces $2y = -kz$ y $y = \dfrac{-kz}{2}$.
		\item[iii] Si $2z + ky = 0$, entonces $z = \dfrac{-ky}{2}$ y por (ii) $z = \dfrac{-k (\dfrac{-kz}{2})}{2} = \dfrac{k^2z}{4}$.
	\end{itemize}
	Además, por (iii) tenemos que $z = \dfrac{k^2z}{4} \Rightarrow \dfrac{z}{z} = \dfrac{k^2}{4} \Rightarrow 1 = \dfrac{k^2}{4} \\ \Rightarrow 4 = k^2 \Rightarrow k = \pm 2$\\
	De (ii) y de (iii) podemos ver que para que se cumpla la igualdad, $y = 0$ y $z = 0$.\\
	Por lo tanto, $f(x,y,z)$ tiene un punto crítico en  $(0,0,0)$.
	\item[b] Cómo $k$ es una constante y el punto crítico de la función es cuándo $x = y = z = 0$, entonces $k \in (-\infty, \infty)$ porque por propiedades de campo, sabemos que $kyz = k(0) = 0$.\\
	Por lo tanto, $k$ puede tomar cualquier valor.\\
<<<<<<< HEAD
=======
\end{itemize}

\textit{23. An examination of the function}
	$$f:\mathbb{R}^2 \mapsto \mathbb{R}, (x, y) \mapsto (y -3x^2)(y -x^2)$$
	\textit{will give an idea of the difficulty of finding conditions that
			guarantee that a critical point is a relative extremum when Theorem
			6 fails. Show that}\\

	\textit{a) the origin is a critical point of $f$ ;}\\

	Procedemos a derivar parcialmente la función, así, se sigue:
	$$f(x,y) = y^2 -3x^2y -x^2y + 3x^4 = y^2 -4x^2y + 3x^4  $$
	$$\frac{\partial f}{\partial x} = -8xy + 12x^3 $$
	$$\frac{\partial f}{\partial y} = 2y - 4x^2 $$

	luego, para ver que precisamente el origen es un punto crítico,
	evaluamos las parciales en dicho punto, así:
	$$\left. \frac{\partial f}{\partial x} \right|_{(0,0)}
			= -8(0)(0) + 12(0)^3 = 0 $$
	$$\left. \frac{\partial f}{\partial y} \right|_{(0,0)}
			= 2(0) - 4(0)^2 = 0 $$

	así, como todas las parciales son cero en el origen, se sigue que este es
	un punto crítico.\\

	\textit{b) $f$ has a relative minimum at $(0, 0)$ on every straight line
			through $(0, 0)$; that is, if $g(t) = (at, bt)$, then
			$f \circ g : \mathbb{R} \rightarrow \mathbb{R}$ has a relative
			minimum at 0,for every choice of a and b;}\\

		Consideremos ahora a la funcíón $f \circ g$:
		$$ f \circ g = (bt)^2 - 4(at)^2(bt) + 3(at)^4$$
		$$ = b^2t^2 - 4a^2bt^3 + 3a^4t^4$$

		ahora, derivando la función compuesta, se tiene:
		$$\frac{\partial f}{\partial t} = 2b^2t -12a^2bt^2 + 12a^4t^3
										= 2t(b^2 - 6a^2bt + 6a^4t^2)$$

		se sigue que $t = 0$ es un punto crítico correspondiente a $(x,y) = (0,0)$.
		Ahora, derivando nuevamente la función, se tiene:

		$$\frac{\partial^2 f}{\partial t^2} = 2b^2 -24a^2bt + 36a^4t^2 $$

		ahora, evaluando la segunda derivada en el punto $t = 0$
		$$\left. \frac{\partial^2 f}{\partial t^2} \right|_{0}
			= 2b^2 -24a^2b(0) + 36a^4(0)^2 = 2b^2 $$

		como
			$$\left. \frac{\partial^2 f}{\partial t^2} \right|_{0} = 2b^2 > 0$$
		se sigue que $t = 0$ es un mínimo local para
		$\forall a$ si $b \neq 0$\\

		Ahora, si $b = 0$, se cumple que
			$$f \circ g (t) = 3(at)^4$$
		así que $t = 0$ resulta ser un mínimo local para cualquier elección de
		$a$ y $b$.\\

	\textit{(c) the origin is not a relative minimum of $f$ .}\\
	\newline

	\textit{\textbf{$27.$}Suppose $f: \mathbb{R}^3 \mapsto \mathbb{R}$ is $C^2$, and that $x_{0}$ is a critical point for $f$. Suppose $Hf (x_{0})(h) = h_{1}^2 + h_{2}^2 + h_{3}^2 + 4h_{2}h_{3}$ Does $f$ have a local maximum, minimum, or saddle a $x_{0}$?}\

\textit 34. Sea $f(x,y) = 5ye^x - e^{5x} -y^5$
\begin{itemize}
	\item[a] Demuestre que  $f$ tiene un único punto crítico que también es máximo de $f$.
	Para encontrar el punto crítico hacemos $\nabla f = 0$. Así,
	\begin{itemize}
		\item[i] $\deriv{f}{x} = 5ye^x-5e^{5x}$. Entonces, $\deriv{f}{x} = 0$ cuando $5ye^x-5e^{5x} = 0$.
		\item[ii]$\deriv{f}{y} = 5e^x-5y^4$. Entonces, $\deriv{f}{y} = 0$ cuando $5e^x-5y^4 = 0$.
	\end{itemize}
	De (i) tenemos entonces que $5e^x = 5e^{5x} \Rightarrow ye^x = e^5x$\\
	De (ii) tenemos que $5e^x = 5y^4 \Rightarrow e^x = y^4$ De esta ecuación, proponemos al vector $\xo = (0,1)$.\\
	Evaluando (ii) en $\xo$ tenemos que
	$$ e^x = e^0 = 1 = 1^4 = y^4$$
	Y, evaluando (i) en $\xo$ tenemos que
	$$ye^x = 1\cdot e^0 = e^0 = 1 = e^{5\cdot 0} = e^{5x}$$
	Por lo tanto, $f(\xo)$ es un punto crítico.
	Para comprobar que $\xo$ es máximo, lo evaluamos en el Hessiano.
	\[
	\begin{bmatrix}
	x & y
	\end{bmatrix}
	\begin{bmatrix}
	    \sderiv{f}{x}{x} & \sderiv{f}{x}{y}\\
		\sderiv{f}{y}{x} & \sderiv{f}{y}{y}\\
	\end{bmatrix}
	\begin{bmatrix}
	x \\ y
	\end{bmatrix}\\
	=
	\begin{bmatrix}
	x & y
	\end{bmatrix}
	\begin{bmatrix}
	    5ye^x - 25e^{5x} & 5e^x\\
		5e^x & -20y^3\\
	\end{bmatrix}
	\begin{bmatrix}
	x \\ y
	\end{bmatrix}\\
	=
	\]
	\[
	\begin{bmatrix}
	x & y
	\end{bmatrix}
	\begin{bmatrix}
	    5ye^x - 25e^{5x} & 5e^x\\
		5e^x & -20y^3\\
	\end{bmatrix}
	\begin{bmatrix}
	x \\ y
	\end{bmatrix}\\
	=
	\begin{bmatrix}
	x & y
	\end{bmatrix}
	\begin{bmatrix}
	    5(1)e^0 - 25e^{5(0)} & 5e^(0)\\
		5e^(0) & -20(1)^3\\
	\end{bmatrix}
	\begin{bmatrix}
	x \\ y
	\end{bmatrix}\\
	=
	\]
	\[
	\begin{bmatrix}
	x & y
	\end{bmatrix}
	\begin{bmatrix}
	    -20 & 5\\
		5 & -20\\
	\end{bmatrix}
	\begin{bmatrix}
	x \\ y
	\end{bmatrix}\\
	=
	\begin{bmatrix}
	-20x + 5y & 5x - 20y
	\end{bmatrix}
	\begin{bmatrix}
	x \\ y
	\end{bmatrix}\\
	=
	\begin{bmatrix}
	-20x^2+5xy+5xy-20y^2
	\end{bmatrix}
	\]
	Entonces, Hess($\xo$) = $[-20x^2+5xy+5xy-20y^2]|_{\xo} = -20(0)+10(0)(1)-20(1) = -20$\\
	Cómo el Hess($\xo$) $< 0$, entonces $\xo$ es un máximo.
	\item[b] Muestre que $f$ no está acotado en el eje $y$ y por lo tanto no tiene máximo global.
	Fijamos a $x = 0$ y evaluamos la funcion $f(0,y)= 5ye^0-e^0-y^5 = 5y-1-y^5$ y observamos que \[ \lim_{x \to \pm \infty} f(0,y) = \pm \infty \]\\
	Por lo tanto, $f(0,y)$ no está acotada y no tiene máximo global.

>>>>>>> d897f1f2aa29d5ee0327e6cbcd9b402096b3e438
\end{itemize}


\textit{37. Determine the nature of the critical points of
			$f (x, y) = x^3 + y^2 - 6xy + 6x + 3y$.}\\

			Procedemos a derivar parcialmente la función para $x, y$ así:

				$$ \frac{\partial f}{\partial x} = 3x^2 - 6y + 6$$
				$$ \frac{\partial f}{\partial y} = 2y -6x + 3$$

			luego, para hallar los extremos locales, igualamos las parciales a cero, así:
				$$ \frac{\partial f}{\partial x} = 3x^2 - 6y + 6 = 0$$
				$$ 3x^2 - 6y + 6 = 0$$
				$$ 2y -6x + 3 = 0$$

			de la segunda ecuación, podemos obtener:
				$$  2y -6x + 3 = 0 $$
				$$  2y = 6x - 3 $$
				$$  y = \frac{6x - 3}{2} $$

			y así, sustituyendo en la primera ecuación:
				$$ 3x^2 - 6y + 6 = 0 $$
				$$ 3x^2 - 6\Big(\frac{6x - 3}{2}\Big) + 6 = 0 $$
				$$ 3x^2 - 18x + 9 + 6 = 0 $$
				$$ 3x^2 - 18x + 15 = 0 $$
				$$ x^2 - 6x + 5 = 0 $$
				$$ (x-1)(x-5) = 0 $$

			de allí obtenemos:

				$$ x = 1 \; x = 5$$

			y haciendo uso se la ecuación $ y = \frac{6x + 3}{2} $, se sigue:
				$$ y = \frac{6(1) - 3}{2} = \frac{3}{2} \; y = \frac{6(5) - 3}{2} = \frac{27}{2} $$

			se sigue que los puntos críticos son:
				$$ \Big( 1, \frac{3}{2} \Big), \Big( 1, \frac{27}{2} \Big)  $$

			luego, procedemos a derivar iteradamente para poder construir la matriz Hessiana
			y así determinar la naturaleza de los puntos críticos, así se sigue:
				$$ \frac{\partial^2 f}{\partial x^2} = 6x $$
				$$ \frac{\partial^2 f}{\partial x \partial y } = -6
				 = \frac{\partial^2 f}{\partial y \partial x } $$
				$$ \frac{\partial^2 f}{\partial y^2} = 2$$

			Así, contruyendo la matriz Hessiana y evaluando en los puntos críticos, se sigue:
				$$H(x,y) = \begin{bmatrix}
						\dfrac{\partial^2f}{\partial x^2} & & \dfrac{\partial^2f}{\partial x\partial y} \\
						& & \\
						\dfrac{\partial^2f}{\partial y\partial x}&  & \dfrac{\partial^2f}{\partial y^2} \\
						\end{bmatrix} =
						\begin{bmatrix}
							6x & -6 \\
							-6  & 2 \\
						\end{bmatrix}
						$$
				$$det \Big[H \Big(1, \frac{3}{2}\Big)\Big] =
						\begin{vmatrix}
							6 & -6 \\
							-6  & 2 \\
						\end{vmatrix} = 12 - 36 = - 24 < 0
				$$
				Como $f_{xx} > 0$ y $det(H) < 0$, entonces $\Big(1, \frac{3}{2}\Big)$ corresponde a un
				punto de ensilladura.\\

				Análogamente:\\
				$$det \Big[H \Big(5, \frac{27}{2}\Big)\Big] =
						\begin{vmatrix}
							30 & -6 \\
							-6  & 2 \\
						\end{vmatrix} = 60 - 36 = 24 > 0
				$$
				Como $f_{xx} > 0$ y $det(H) > 0$, entonces $\Big(5, \frac{27}{2}\Big)$
				corresponde a un mínimo local.\\

<<<<<<< HEAD
	\textit{(c) the origin is not a relative minimum of $f$ .}\\
	\newline


\textit{\textbf{$27.$}Suppose $f: \mathbb{R}^3 \mapsto \mathbb{R}$ is $C^2$, and that $x_{0}$ is a critical point for $f$. Suppose $Hf (x_{0})(h) = h_{1}^2 + h_{2}^2 + h_{3}^2 + 4h_{2}h_{3}$ Does $f$ have a local maximum, minimum, or saddle a $x_{0}$?}\\

Tenemos que $Hf(x_0)(h) = h_1^2+h_2^2+h_3^2++4h_2  h_3$, si factorizamos, entonces:
\[Hf(x_0)(h)= h_1^2+(h_2+h_3)^2+2h_2h_3\]
\[g \left( \dfrac{h}{ \parallel h \parallel}\right). \parallel h \parallel^2\]
donde $\norm{h} ^2 = h_1^2+h_2^2+h_3^2$ y $g $ una función. $H(h)\geq 1. \norm{h}^2$ entonces, dependiendo de $h_2, h_3$  $H(h)$ puede ser positivo o negativo y así tener un local mínimo o máximo o un punto de ensilladura.
Sea $h_1=h_2=h_3$, luego $H(h)= 7h^2$ siendo positivo (mínimo local)
$h_1=h_3= h$, $h_2 = -h$, luego $H(h) = -h^2$ siendo negativo (máximo local).

Por lo tanto, el punto críto es un punto de ensilladura, en una dirección tiene mínimo y en otra tiene un máximo.\\

\textit 34. Sea $f(x,y) = 5ye^x - e^{5x} -y^5$
\begin{itemize}
	\item[a] Demuestre que  $f$ tiene un único punto crítico que también es máximo de $f$.
	Para encontrar el punto crítico hacemos $\nabla f = 0$. Así,
	\begin{itemize}
		\item[i] $\deriv{f}{x} = 5ye^x-5e^{5x}$. Entonces, $\deriv{f}{x} = 0$ cuando $5ye^x-5e^{5x} = 0$.
		\item[ii]$\deriv{f}{y} = 5e^x-5y^4$. Entonces, $\deriv{f}{y} = 0$ cuando $5e^x-5y^4 = 0$.
	\end{itemize}
	De (i) tenemos entonces que $5e^x = 5e^{5x} \Rightarrow ye^x = e^5x$\\
	De (ii) tenemos que $5e^x = 5y^4 \Rightarrow e^x = y^4$ De esta ecuación, proponemos al vector $\xo = (0,1)$.\\
	Evaluando (ii) en $\xo$ tenemos que
	$$ e^x = e^0 = 1 = 1^4 = y^4$$
	Y, evaluando (i) en $\xo$ tenemos que
	$$ye^x = 1\cdot e^0 = e^0 = 1 = e^{5\cdot 0} = e^{5x}$$
	Por lo tanto, $f(\xo)$ es un punto crítico.
	Para comprobar que $\xo$ es máximo, sacamos el determinante de su matriz Hessiana.
	\[
	\begin{bmatrix}
	    \sderiv{f}{x}{x} & \sderiv{f}{x}{y}\\
		\sderiv{f}{y}{x} & \sderiv{f}{y}{y}\\
	\end{bmatrix}\\
	=
	\begin{bmatrix}
	    5ye^x - 25e^{5x} & 5e^x\\
		5e^x & -20y^3\\
	\end{bmatrix}\\
	=
	\]
	\[
	\begin{bmatrix}
	    5ye^x - 25e^{5x} & 5e^x\\
		5e^x & -20y^3\\
	\end{bmatrix}\\
	=
	\begin{bmatrix}
	    5(1)e^0 - 25e^{5(0)} & 5e^(0)\\
		5e^(0) & -20(1)^3\\
	\end{bmatrix}\\
	=
	\begin{bmatrix}
	    -20 & 5\\
		5 & -20\\
	\end{bmatrix}\\
	\]
	De aquí podemos ver que det(Hess($\xo$)) = $(-20)(-20) - 5*5= 200 -25 = 175$\\
	Entonces, cómo det( Hess($\xo$)) $>0$ y $\sderiv{f}{x}{x}(\xo) < 0 $, la función tine un máximo local en $\xo= (0,1)$.
	\item[b] Muestre que $f$ no está acotado en el eje $y$ y por lo tanto no tiene máximo global.
	Fijamos a $x = 0$ y evaluamos la funcion $f(0,y)= 5ye^0-e^0-y^5 = 5y-1-y^5$ y observamos que \[ \lim_{x \to \pm \infty} f(0,y) = \pm \infty \]\\
	Por lo tanto, $f(0,y)$ no está acotada y no tiene máximo global.

\end{itemize}


\textit{37. Determine the nature of the critical points of
			$f (x, y) = x^3 + y^2 - 6xy + 6x + 3y$.}\\

			Procedemos a derivar parcialmente la función para $x, y$ así:

				$$ \frac{\partial f}{\partial x} = 3x^2 - 6y + 6$$
				$$ \frac{\partial f}{\partial y} = 2y -6x + 3$$

			luego, para hallar los extremos locales, igualamos las parciales a cero, así:
				$$ \frac{\partial f}{\partial x} = 3x^2 - 6y + 6 = 0$$
				$$ 3x^2 - 6y + 6 = 0$$
				$$ 2y -6x + 3 = 0$$

			de la segunda ecuación, podemos obtener:
				$$  2y -6x + 3 = 0 $$
				$$  2y = 6x - 3 $$
				$$  y = \frac{6x - 3}{2} $$

			y así, sustituyendo en la primera ecuación:
				$$ 3x^2 - 6y + 6 = 0 $$
				$$ 3x^2 - 6\Big(\frac{6x - 3}{2}\Big) + 6 = 0 $$
				$$ 3x^2 - 18x + 9 + 6 = 0 $$
				$$ 3x^2 - 18x + 15 = 0 $$
				$$ x^2 - 6x + 5 = 0 $$
				$$ (x-1)(x-5) = 0 $$

			de allí obtenemos:

				$$ x = 1 \; x = 5$$

			y haciendo uso se la ecuación $ y = \frac{6x + 3}{2} $, se sigue:
				$$ y = \frac{6(1) - 3}{2} = \frac{3}{2} \; y = \frac{6(5) - 3}{2} = \frac{27}{2} $$

			se sigue que los puntos críticos son:
				$$ \Big( 1, \frac{3}{2} \Big), \Big( 1, \frac{27}{2} \Big)  $$

			luego, procedemos a derivar iteradamente para poder construir la matriz Hessiana
			y así determinar la naturaleza de los puntos críticos, así se sigue:
				$$ \frac{\partial^2 f}{\partial x^2} = 6x $$
				$$ \frac{\partial^2 f}{\partial x \partial y } = -6
				 = \frac{\partial^2 f}{\partial y \partial x } $$
				$$ \frac{\partial^2 f}{\partial y^2} = 2$$

			Así, contruyendo la matriz Hessiana y evaluando en los puntos críticos, se sigue:
				$$H(x,y) = \begin{bmatrix}
						\dfrac{\partial^2f}{\partial x^2} & & \dfrac{\partial^2f}{\partial x\partial y} \\
						& & \\
						\dfrac{\partial^2f}{\partial y\partial x}&  & \dfrac{\partial^2f}{\partial y^2} \\
						\end{bmatrix} =
						\begin{bmatrix}
							6x & -6 \\
							-6  & 2 \\
						\end{bmatrix}
						$$
				$$det \Big[H \Big(1, \frac{3}{2}\Big)\Big] =
						\begin{vmatrix}
							6 & -6 \\
							-6  & 2 \\
						\end{vmatrix} = 12 - 36 = - 24 < 0
				$$
				Como $f_{xx} > 0$ y $det(H) < 0$, entonces $\Big(1, \frac{3}{2}\Big)$ corresponde a un
				punto de ensilladura.\\

				Análogamente:\\
				$$det \Big[H \Big(5, \frac{27}{2}\Big)\Big] =
						\begin{vmatrix}
							30 & -6 \\
							-6  & 2 \\
						\end{vmatrix} = 60 - 36 = 24 > 0
				$$
				Como $f_{xx} > 0$ y $det(H) > 0$, entonces $\Big(5, \frac{27}{2}\Big)$
				corresponde a un mínimo local.\\

=======
>>>>>>> d897f1f2aa29d5ee0327e6cbcd9b402096b3e438


	\textit{41. Find the absolute maximum and minimum values for
		$$f(x, y) = sin x + cos y$$ on the rectangle $R$ defined by
		$$0 \leq x \leq 2\pi, 0 \leq y \leq 2\pi$$}

		Derivando parcialmente, se tiene:
		$$\frac{\partial f}{\partial x} = \cos x $$
		$$\frac{\partial f}{\partial y} = - \sin x $$

		ahora, para los intervalos $0 \leq x \leq 2\pi, 0 \leq y \leq 2\pi$
		se cumple:
			$$\frac{\partial f}{\partial x} = 0 \quad \text{si} \quad
				x=\frac{\pi}{2}, \frac{3\pi}{2}$$
			$$\frac{\partial f}{\partial y} = 0 \quad \text{si} \quad
				y=0, \pi, 2\pi$$

		así, se sigue que la función tiene puntos críticos en:
			$$(\frac{\pi}{2}, 0), (\frac{\pi}{2}, \pi), (\frac{\pi}{2}, 2\pi)$$
			$$(\frac{3\pi}{2}, 0), (\frac{3\pi}{2}, \pi), (\frac{3\pi}{2}, 2\pi)$$

		ahora, construimos la matriz Hessiana para la función derivando
		parcialmente por segunda vez a la función, así:
			$$ \frac{\partial^2f}{\partial x^2} = -\sin x$$
			$$ \frac{\partial^2f}{\partial x\partial y} = 0 = \frac{\partial^2f}{\partial y\partial x}$$
			$$ \frac{\partial^2f}{\partial y^2} = - \cos y$$

			$$H(x,y) = \begin{bmatrix}
    					\dfrac{\partial^2f}{\partial x^2} & & \dfrac{\partial^2f}{\partial x\partial y} \\
    					& & \\
    					\dfrac{\partial^2f}{\partial y\partial x}&  & \dfrac{\partial^2f}{\partial y^2} \\
						\end{bmatrix} =
						\begin{bmatrix}
			    			-\sin x & 0 \\
			    			0  & - \cos y \\
						\end{bmatrix}
						$$

		luego, evaluando la matriz en los puntos críticos y sacando su determinante,
		se tiene:
		$$det[H(\frac{\pi}{2}, 0)] =
			\begin{vmatrix}
				-\sin \dfrac{\pi}{2} & 0 \\
				0  & - \cos 0 \\
			\end{vmatrix} =
			\begin{vmatrix}
				-1 & 0 \\
				 0 & -1 \\
			\end{vmatrix} = 1$$

		Como $f_{xx} < 0$ y $det(H) > 0$, entonces corresponde a un máximo local.\\
		$$det[H(\frac{\pi}{2}, \pi)] =
			\begin{vmatrix}
				-\sin \dfrac{\pi}{2} & 0 \\
				0  & - \cos \pi \\
			\end{vmatrix} =
			\begin{vmatrix}
				-1 & 0 \\
				 0 & 1 \\
			\end{vmatrix} = -1$$
		Como $f_{xx} < 0$ y $det(H) < 0$, entonces corresponde a un punto de ensilladura.\\
		$$det[H(\frac{\pi}{2}, 2\pi)] =
			\begin{vmatrix}
				-\sin \dfrac{\pi}{2} & 0 \\
				0  & - \cos 2\pi \\
			\end{vmatrix} =
			\begin{vmatrix}
				-1 & 0 \\
				 0 & -1 \\
			\end{vmatrix} = 1$$
		Como $f_{xx} < 0$ y $det(H) > 0$, entonces corresponde a un máximo local.\\


		$$det[H(\frac{3\pi}{2}, 0)] =
			\begin{vmatrix}
				-\sin \dfrac{3\pi}{2} & 0 \\
				0  & - \cos 0 \\
			\end{vmatrix} =
			\begin{vmatrix}
				1 & 0 \\
				 0 & -1 \\
			\end{vmatrix} = -1$$

		Como $f_{xx} > 0$ y $det(H) < 0$, entonces corresponde a un punto de ensilladura.\\
		$$det[H(\frac{3\pi}{2}, \pi)] =
			\begin{vmatrix}
				-\sin \dfrac{3\pi}{2} & 0 \\
				0  & - \cos \pi \\
			\end{vmatrix} =
			\begin{vmatrix}
				1 & 0 \\
				 0 & 1 \\
			\end{vmatrix} = 1$$
		Como $f_{xx} > 0$ y $det(H) > 0$, entonces corresponde a un mínimo local.\\
		$$det[H(\frac{3\pi}{2}, 2\pi)] =
			\begin{vmatrix}
				-\sin \dfrac{3\pi}{2} & 0 \\
				0  & - \cos 2\pi \\
			\end{vmatrix} =
			\begin{vmatrix}
				1 & 0 \\
				 0 & -1 \\
			\end{vmatrix} = -1$$
		Como $f_{xx} > 0$ y $det(H) < 0$, entonces corresponde a un punto de ensilladura.\\


		Luego, evaluando máximos locales en la función para determinar
		máximos absolutos:
			$$ f(\frac{\pi}{2}, 0) = \sen(\frac{\pi}{2}) + cos(\pi) = 1 + 1 = 2 $$
			$$ f(\frac{\pi}{2}, 2\pi) = \sen(\frac{\pi}{2}) + cos(2\pi) = 1 + 1 = 2$$

		así, determinamos que la función $f$ tiene máximos absolutos en los puntos
			$$(\frac{\pi}{2}, 0), (\frac{\pi}{2}, 2\pi)$$
		y mínimo absoluto en
			$$(\frac{3\pi}{2}, \pi)$$

<<<<<<< HEAD
\textit{\textbf{51.}}
	\begin{itemize}
		\item[(a)] Let f be a $C^1$ function on the real line $\mathbb{R}$. Suppose that f has exactly one critical point $x_{0}$ that is a strict local minimum for $f$ . Show that $x_{0}$ is also an absolute minimum for $f$; that is, that $f(x) \geq f(x_{0})$ for all $x$.

		Como $f$ tiene exactamente un punto crítico en $x_{0}$ y es una función de $C'$, entonces :
		$f'(x) = 0$ para $x= x_{0}$ y $f'(x)\neq 0$ para $x\neq x_{0}$
		$f$ es continua en la recta real, entonces, ya sea que $f$ aumente o decremente en cualquier punto $x\neq x_{0}$ . Como $x_{0}$ es un mínimo local, tenemos que :
		$f(x_0)< f(x)$ para todo $0<| x-x_{0}| < \delta$, entonces para $x> x_{0}$ la derivada de $x$ será mayor a cero y para $x<x_{0}$ la derivada de $f$ evaluada en $x$ será menor a cero.
		Luego, $f$ está decreciendo para toda $x$ tal que $x<x_{0}$, así $f(x)\geq f(x_0)$ para toda $x$ que pertenece a los reales y $x_0$ es también un mínimo absoluto de $f$
		\item[(b)] The next example shows that the conclusion of part (a) does not hold for functions of more tha one variable. Let $f: \mathbb{R}^2 \mapsto \mathbb{R}$ be defined by
		\[f(x,y) = -y^4-e^{-x^2}+2y^2 \sqrt{e^x + e^{-x^2}}\]
		show that $(0,0)$ is the only critical point for $f$ and that is a local minimum
		\[\bigtriangledown f = \left[ 2xe^{-x^2}+2y^2 \dfrac{\left( e^x -2xe^{-x^2} \right)}{2\sqrt{e^x+e^{-x^2}}}\right]i +\left[-4y^3+4y \sqrt{e^x+e^{-x^2}} \right]j\]
		Ahora $\dfrac{\delta f}{\delta x} = 0$, entonces :
		\[2xe^{-x^2}+ \dfrac{2y^2 \left(e^x-ex e^{-x^2} \right)}{2\sqrt{e^x + e^{-x^2}}}= 0\]
		\[y^2(e^x-2x e^{-x^2})= -2xe^{-x^2}\sqrt{e^x+ e^{-x^2}}\]
		\[y^4 \left(e^{2x}+4x^2 e^{x-x^2} \right) = 4x^2e^{-2x^2}\left(e^x + e^{-x^2}\right)\]
	$(1):$
		\[y^4\left(e^{2x}+4x^2e^{-2x^2}-4xe^{x-x^2} \right)+ 4x^2e^{-2x^2}\left( e^x+ e^{-x^2}\right)=0 \]
		También $\dfrac{\delta f}{\delta y} = 0$, entonces $4y\left(-y^2+ \sqrt{e^x + e^{-x^2}}\right)=0$. Así $y=0$ o $(2):$
		\[y^2= \sqrt{e^x+ e^{-x^2}}\]
		Cuando $y = 0$ de $(1)$
		\[2x \left(e^{-x^2+x}+ e^{-2x^2} \right)=0\]
		Entonces $x=0$, así que un punto crítico es $(0,0)$
		Ahora, cuando $y = \sqrt{e^x+ e^{-x^2}}$, de $(1)$ tenemos que :
		\[(e^x+ e^{-x^2})(e^{2x}+4x^2e^{-2x^2}-4xe^{x-x^2})+2xe^{-x^2}(e^x+ e^{-x^2})=0\]
		Ó :
		\[(e^x+e^{-x^2})x(e^x-2xe^{-x^2})^2 +2xe^{-x^2}=0\]
		Ó :
		\[2x=\dfrac{-\left( e^x- 2xe^{-x^2}\right)^2}{2e^{-x^2}}\]
		y esta última no cumple para algún $x\epsilon R$, así que el único punto crítico es $(0,0)$
		Para ver que es un mínimo local :
		\[\dfrac{\delta^2 f}{\delta x^2}=\dfrac{-4x^2e^{-x^2}+2e^{-x^2}+y^2 \left[ \left( e^x+ 4x^2-2e^{-x^2}\right)x \sqrt{e^x+e^{-x^2}}- \dfrac{\left(e^x- 2xe^{-x^2} \right)^2}{2 \sqrt{e^x+ e^{-x^2}}} \right]}{\left(e^x+ e^{-x^2} \right)}\]
		entonces : $\dfrac{\delta ^2 f}{\delta y^2}(0,0) = 2 $ y también $\dfrac{\delta ^2 f}{\delta y^2}= -12y^2+ 8 \sqrt{e^x + e^-x^2}$
		y $\dfrac{\delta ^2 f}{\delta y^2}(0,0)= 8 \sqrt{2}$ Y $\dfrac{\delta^2 f}{\delta y \delta x}= 2y \dfrac{(e^x- 2xe^{-x^2})}{\sqrt{e^x+ e^{-x^2}}}$, también $\dfrac{\delta^2 f(0,0)}{\delta y \delta x}= 0$, ahora :
		\[\left(\dfrac{\delta^2 f}{\delta y \delta x}(0,0) \right)^2 - \dfrac{\delta^2 f}{\delta x^2}(0,0).\dfrac{\delta ^2 f}{\delta y^2}(0,0) = -16\sqrt{2}<0\]
		También :
		\[\dfrac{\delta ^2 f}{\delta x^2}(0,0)+\dfrac{\delta ^2 f}{\delta y^2}(0,0)= 2+8\sqrt{2}> 0\]
		Por lo tanto, $(0,0)$ es un punto mínimo local.
	\end{itemize}


	\textit{\textbf{52} Suponga que un pentágono está compuesto de un rectángulo con un triángulo isóceles encima. Si la longitud del perímetro está fija, encuentre la máxima área posible.}\\
	Sean $z = \dfrac{y}{2cos\theta}$ el tamaño de cada uno de los lados del triángulo que cubre al rectángulo que forma al pentágono y $w = zsen\theta = \dfrac{y sen\theta}{2cos\theta}$ la altura de dicho triángulo. \\
	El perímetro $p = 2x+y+2z=2x+y+\dfrac{y}{cos\theta}$, p es constante.\\
	El área $A=xy+2(\dfrac{1}{2}w\dfrac{y}{2})=xy+w\dfrac{y}{2}= xy + z sen\theta.$\\
	Así, despejando a $x$ de la restricción, tenemos que $p = 2x +y+\dfrac{y}{cos\theta} \Rightarrow x = \dfrac{1}{2}(p-y-\dfrac{y}{cos\theta})$. Entonces, $A=\dfrac{1}{2}y(p-y-\dfrac{y}{cos\theta})+\dfrac{1}{2}yz\sen\theta = \dfrac{1}{2}(py-y^2-\dfrac{y^2}{cos\theta}+\dfrac{y^2sen\theta}{2cos\theta}$. De esta manera,
	$$\deriv{A}{y} = \dfrac{p}{2}-y-\dfrac{y}{cos\theta}+ \dfrac{ysen\theta}{2cos\theta} \ \ (i)$$
	$$\deriv{A}{\theta} = -\dfrac{y^2}{2} - \dfrac{sen\theta}{cos^2\theta}+\dfrac{y^2}{4cos^2\theta} \ \ (ii)$$
	Al hacer $(ii) = 0$ tenemos que
	$$\dfrac{y^2}{4cos^2\theta} = \dfrac{y^2}{2} + \dfrac{sen\theta}{cos^2\theta} \Rightarrow \dfrac{1}{2} = sen\theta \Rightarrow \theta= \dfrac{\pi}{6} \Rightarrow cos\theta = \dfrac{\sqrt{3}}{2}\ \ (iii)$$
	De esta manera, tenemos entonces por $(i)$ y por $(iii)$ que $$\dfrac{p}{2}= y(1+\dfrac{2}{\sqrt{3}}-\dfrac{1}{2\sqrt{3}})\Rightarrow $$$$y = \dfrac{p}{2(1+\dfrac{2}{\sqrt{3}}-\dfrac{1}{2\sqrt{3}})}=\dfrac{p}{2+\dfrac{4}{\sqrt{3}}-\dfrac{1}{\sqrt{3}}}=\dfrac{p\sqrt{3}}{2\sqrt{3}+4-1}= \dfrac{p\sqrt{3}}{2\sqrt{3}-3}$$
	Por lo tanto, $y=\dfrac{p\sqrt{3}}{2\sqrt{3}-3}$ es la máxima area posible dado un perímetro fijo.


=======
	\textit{\textbf{51.}
		\begin{itemize}
			\item[(a)] Let f be a $C^1$ function on the real line $\mathbb{R}$. Suppose that f has exactly one critical point $x_{0}$ that is a strict local minimum for $f$ . Show that $x_{0}4$ is also an absolute minimum for $f$; that is, that $f(x) \geq f(x_{0})$ for all $x$.
			\item[(b)] The next example shows that the conclusion of part (a) does not hold for functions of more tha one variable. Let $f: \mathbb{R}^2 \mapsto \mathbb{R}$ be defined by
			\[f(x,y) = -y^4-e^{-x^2}+2y^2 \sqrt{e^x + e^{-x^2}}\]
		\end{itemize}
	}


>>>>>>> d897f1f2aa29d5ee0327e6cbcd9b402096b3e438
	\section{Marsden - Tromba | Sección 3.4}

\textit{\textbf{26.}An irrigation canal in Arizona has concrete sides and bottom with trapezoidal cross section of area
 $A = y(x + y \tan \theta)$ and wetted perimeter \\
 $P = x + \dfrac{2y}{\cos \theta}$, where $x = $bottom width, $y = $water depth, and $\theta =$side inclination, measured
 from vertical.The best design for a fixed inclination $\theta$ is found by solving $P =$ minimum subject to the condition
 $A =$ constant. Show that $y^2 =\dfrac{A \cos \theta}{2 - \sin \theta}$
}

Tenemos que $f(x,y, \lambda) = x+ 2y \sec \theta + \lambda(xy+ y^2 tan\theta - A)$
Ahora $\dfrac{\delta f}{\delta x}= 1+ \lambda y = 0$, entonces $\lambda = \dfrac{-1}{y}$
\[\dfrac{\delta f}{\delta y} = 2 \sec \theta + \lambda(x + 2y\tan \theta)= 0\]
\[\lambda = \dfrac{-2 \sec\theta}{x+2y \tan \theta}\]
y
\[\dfrac{\delta f}{\delta \lambda} = xy+ y^2 \tan \theta - A =0 \Longleftrightarrow A = xy+ y^2 \tan \theta\]
Igualamos los valores de $\lambda$, entonces tenemos :
\[2y = \dfrac{x+ 2y \tan \theta}{\sec \theta}= \dfrac{x\cos \theta+ 2y \sin \theta}{1}\]
\[\Rightarrow 2y(1-\sin \theta) = x \cos \theta\]
\[\Rightarrow x = \dfrac{2y(1-\sin \theta)}{\cos \theta}= 2y (\sec \theta - \tan \theta)\]
Susstituyendo dentro $\dfrac{\delta f}{\delta \lambda} = 0$ tenemos que :
\[A = 2y^2(\sec \theta - \tan \theta)+ y^2\tan \theta = y^2 (2 \sec\theta - \tan \theta)\]
\[\Rightarrow y^2 = \dfrac{A}{2\sec \theta - \tan \theta}\]
\[y^2 = \dfrac{A \cos \theta}{2- \sin \theta}\]

\textit{\textbf{26.} Sea $A$ una matríz $3x3$ simétrica. Considere la función $f(x)=\dfrac{1}{2}(A\cdot x)x$}
\begin{itemize}
	\item[a] ¿Cual es $\nabla f$?\\
	$$f(x) = \dfrac{1}{2}(a_{11}x^2+a_{12}xy+a_{13}xz+a_{21}yx+a_{22}y^2+a_{23}yz+a_{31}xz+a_{32}zy+a_{33}z^2)$$
	Sabemos que $\nabla f = \big(\deriv{f}{x},\deriv{f}{y},\deriv{f}{z}\big)$
	\begin{align}
		\deriv{f}{x}&=\dfrac{1}{2}(2a_{11}x+2a_{12}y+2a_{13}z)=(a_{11}x+a_{12}y+a_{13}z)\\
		\deriv{f}{y}&=\dfrac{1}{2}(2a_{21}x+2a_{22}y+2a_{23}z)=(a_{21}x+a_{22}y+a_{23}z)\\
		\deriv{f}{z}&=\dfrac{1}{2}(2a_{31}x+2a_{32}y+2a_{33}z)=(a_{31}x+a_{32}y+a_{33}z)
	\end{align}
	Entonces, $$\nabla f =(a_{11}x+a_{12}y+a_{13}z,a_{21}x+a_{22}y+a_{23}z,a_{31}x+a_{32}y+a_{33}z)=Ax$$
	Por lo tanto, $\nabla f = Ax$
	\item[b] Consierando la restricción de $f$ en la esfera unitaria $S={(x, y, z)|x^2 + y^2 + z^2 = 1}$ en $\mathbb{R}^3$. Por el teorema $7$ sabemos que $f$ debe tener un máximo y un mínimo en $S$. Demuestre que debe de haber un eigenvalor $\lambda$ tal que $Ax = \lambda x$
\end{itemize}
	\section{Hughes-Hallet | Sección 16.1}
	\textit{\textbf{5.} Figure $16.8$ shows a contour plot of population density, people per square kilometer, in a rectangle of land $3$ km by $2$ km. Estimate the population in the refion represented by Figure $16.8$}

Sea R el rectángulo y $f(x,y)$ la densidad poblacional en el punto $(x,y)$, la población está dada por $\int_{R}f(x,y)dA$.. Si aproximamos la integral con sumas de Riemann usando la información de cada km por km, entonces podemos tomar la densidad en un un punto y sumarlos.
Vamos a dividir el eje $y$ en dos partes y el eje $x$ en tres, tenemos los puntos:
(0.5, 1.5)=500, (0.5, 0.5)= 650, (1.5,1.5)= 450, (1.5, 0.5)=200, (2.5, 1.5)=100 y (2.5, 0.5)= 400
por lo que la densidad por km cuadrado es la suma de estos valores
\[\int_{R}(x,y)dA\approx 500 + 450 + 100 + 650 + 200 + 400 = 2300\]

\textit{\textbf{6-12} Decide(without calculation) wheter the integrals are positive, negative, or zero. Let $D$ be the region inside the unit circle centered at the origin, let $R$ be the rigth half of $D$ and let $B$ be the bottom half of $D$}
\begin{itemize}
	\item[6.]$\int_{D}dA$
	         Es positiva, la integral de $f(x,y) = 1$
	\item[7.]$\int_{R}5xdA$
			  La integral es positiva, tomemos en cuenta que es la mitad derecha del área total del círculo unitario, entonces toma x toma valores mayores a $0$. La función siendo integrada es $f(x,y) = 5x$
	\item[8.]$\int_{B}5xdA$
			  B es una función simétrica respecto a $x$ por lo que la integral se cancela: $f(x,y) = -f (-x,y)$ y el resutado es cero.
	\item[9.]$\int_{D}(y^3 + y^5)dA$
		  	   D es simétrico respecto a $y$, por lo que la integral es cero también
	\item[10.]$\int_{B}(y^3+y^5)dA$
			   Los valores de la función con $y < 0$ (porque es la mitad de abajo del círculo)son negativos,por lo que la integral también lo es.
	\item[11.]$\int_{D}(y-y^3)dA$
				$f(x,y) = y- y^3$ es una función impar, y D es simétrica respecto a $y$ por lo que la integral se cancela y es igual a cero
	\item[12.]$\int_{B}(y-y^3)dA$
			 La función $f(x,y) = y -y^3$ es negativa en la región B ya que todos los valores de $y$ en esa región son negativos, entonces tenemos que $\mid y^3 \mid < \mid y \mid$ por lo que la integral es negativa.
	\item[15.] \textit{Figure 16.10 shows the temperature,
			   in ºC, in a 5 meter by 5 meter heated room. Using Riemann sums,
			   estimate the average temperature in the room.}\\

	Tomando como punto de referencia el extremo inferior izquierdo de cada uno
	de los rectángulos, podemos expresar a la máxima temperatura en un cierto
	cuadro mediante la siguiente suma de Riemann (consideramos $\Delta A = 1$,
	pues los cambios en $x^{*}_{k}$ y $y^{*}_{k}$ se dan cada unidad):

		$$ \sum_{k=1}^{25} f(x^{*}_{k}, y^{*}_{k}) \Delta A_{k} =
		   	 31(1) + 29(1) + 28(1) + 27(1) + 27(1)  $$
		$$ + 29(1) + 28(1) + 27(1) + 27(1) + 26(1) $$
		$$ + 28(1) + 27(1) + 26(1) + 26(1) + 26(1) $$
		$$ + 26(1) + 26(1) + 25(1) + 25(1) + 25(1) $$
		$$ + 25(1) + 25(1) + 24(1) + 24(1) + 24(1) $$

		$$ = (1) [31 + 29 + 28 + 27 + 27  $$
   		$$ + 29 + 28 + 27 + 27 + 26 $$
   		$$ + 28 + 27 + 26 + 26 + 26 $$
   		$$ + 26 + 26 + 25 + 25 + 25 $$
   		$$ + 25 + 25 + 24 + 24 + 24]  = 661$$

	Análogamente, para calcular la mínima temperatura en un cierto rectángulo,
	tomamos la siguiente suma de Riemann, considerando el extremo superior
	derecho de cada rectángulo:

	$$ \sum_{k=1}^{25} f(x^{*}_{k}, y^{*}_{k}) \Delta A_{k} =
		 27(1) + 26(1) + 26(1) + 25(1) + 25(1)  $$
	$$ + 26(1) + 25(1) + 25(1) + 25(1) + 25(1) $$
	$$ + 25(1) + 24(1) + 24(1) + 24(1) + 24(1) $$
	$$ + 24(1) + 23(1) + 23(1) + 23(1) + 23(1) $$
	$$ + 23(1) + 21(1) + 21(1) + 22(1) + 23(1) $$

	$$ = (1) [27 + 26 + 26 + 25 + 25  $$
	$$ + 26 + 25 + 25 + 25 + 25 $$
	$$ + 25 + 24 + 24 + 24 + 24 $$
	$$ + 24 + 23 + 23 + 23 + 23 $$
	$$ + 23 + 21 + 21 + 22 + 23] = 602 $$

	Así, promediando las temperturas extremas en cada cuadro, obtenemos la temperatura
	promedio del cuarto, así:
		$$ \frac{661 + 602}{2} = 631.5$$
<<<<<<< HEAD
\end{itemize}
\textit{\textbf{21-30} Determine si las siguientes declaraciones son verdaderas o falsas. Justifique.}
\begin{itemize}
	\item[21] Falso. La doble integral de una función que tiene valores $< 0 $ para cualquier $(x,y)$ es negativa.
	\item[22] Verdadero. Por la definición de doble integral, sabemos que podemos sacar constantes de la integral y el integrando va a ser el mismo.
	\item[23] Falso. Evaluamos la función en su punto más grande que es $(1,1)$ y obtenemos que $e^{xy}=e^{1\cdot1}=e^1=e$. Por teorema sabemos que el valor promedio de la función es menor o igual al valor más grande, pero cómo el valor más grande es $e$ y $e < 3$, ell promedio de la función es menor que tres.
	\item[24] Falso. Si $f(x,y) = 1$, el valor de la integral en el intervalo $R$ es $6$ y en el intervalo $S$ es $6$ también.
	\item[25] Verdadero. Por definición, estamos buscando la densidad, es decir el preomedio, en paralelepipeditos cada vez más pequeños de "personas" en el área dada. Entonces, en el límite de los paralelepipeditos (cuando son más pequeños) obtenemos el total de personas en el área.
	\item[26] Falso. Sea $f(x,y)$ tal que existe un valor distinto de cero para esta función. Si la integramos en un punto, el área es cero pero su evaluación no lo es.
	\item[27] Verdadero. Por la definición de integral (sumas de Riemann) podemos sacar cualquier constante de la suma y obtener el mismo resultado. Por lo tanto podemos sacar la constante de la integral.
	\item[28] Falso. Contraejemplo: sean $f,g=1$ y $dA=2$ por un lado tenemos que \[\int_Rf\cdot g dA= \int_R 1\cdot 1 dA = \int_R 1 dA = 2\]\\
	Por otro lado, \[\int_R f dA = \int_R dA = \int_R g dA\] Pero habíamos definido \[\int_R dA = 2\] Entonces, \[\int_R f dA \cdot \int_R g dA = 2 \cdot 2 = 4\] pero $2 \neq 4$.\\
	Por lo tanto, \[\int_Rf\cdot g dA \neq \int_R f dA \cdot \int_R g dA\]
	\item[29] Falso. Si el rectángulo es de un tamaño diferente al doble del tamaño del cuadrado, su área va a ser diferente al doble del área del cuadrado.
	\item[30] Verdadero. Cómo $x < y$, por propiedades de campo $2x<x+y$. Entonces, el área debajo de $g$ siempre va a ser mayor que el área bajo $f$. Esto se puede ver claramente en la gráfica de la función.
\end{itemize}
	\section{Anton-Bivens-Davis | Sección 14.1}
	\textit{\textbf{10.} $\int^{\pi}_{\pi / 2}\int^2_1 x \cos xy dy dx$}\\

Primero resolvemos la integral $\int^2_1 x \cos xy dy$ con cambio de variable. Sea $u= xy$, entonces si derivamos con respecto a $y$,  $du = x dy$, por lo que la integral ahora es
\[\int^2_1 \cos u du = \sin u\mid ^2 _1\]
\[= \sin(xy) \mid^{y=2}_{y=1}\]
Y ahora resolvemos la doble integral sustituyendo :
\[\int^\pi_{\pi / 2} \sin(xy) \mid^{y=2}_{y=1} dx\]
\[\int^\pi_{pi / 2} \sin(2x)-\sin(x) dx\]
Así:
\[\int^\pi_{pi / 2} \sin(2x) dx - \int^\pi_{pi / 2} \sin(x) dx\]
Para la primera integral, definimos a $u$ como $u = 2x$ entonces $du = 2 dx$, luego $dx = 1/2 du$
\[1/2 \int^\pi_{pi / 2}\sin(u) du - \int^\pi_{pi / 2} \sin(x) dx\]
\[1/2(-\cos(2x)\mid^\pi_{\pi / 2})-(-\cos(x)\mid^\pi_{\pi / 2}) \]
\[= 1/2[-\cos(2\pi)+cos(\pi)]+ [\cos(\pi)-\cos(\pi / 2)] \]
Evaluamos :
\[= 1/2(-1-1)+(-1+0) = -2\]
Entonces:
\[\int^{\pi}_{\pi / 2}\int^2_1 x \cos(xy) dy dx = -2\]

	\textit{textbf{16} En la hoja de cálculo podemos observar cómo al integrar la función, es irrelevante si integramos primero en función de $x$ y luego en función de $y$ que hacerlo al revés.}

	\textit{\textbf{18.} (a) Let $f(x,y) = x - 2y$,and as shown in Exercise 17,
	let the rectangle $ R = [0, 2] x [0, 2]$ be subdivided into 16 subrectangles.\\
	\newline
	Take $(x^{*}_{k}, y^{*}_{k})$ to be the center of the $k-th$ rectangle, and approximate
	the double integral of f over R by the resulting Riemann sum.}\\

	Se tiene, que como $(x^{*}_{k}, y^{*}_{k})$ corresponden a los puntos medios
	de cada uno de los $k$ rectángulos, es necesario calcular el valor de cada
	uno de los puntos bajo la función $f$, así, como el rectángulo R está
	comprendido entre

		$$ 0 \leq x \leq 2, 0 \leq y \leq 2  $$

	y dicho rectángulo se divide en 16 rectángulos de igual tamaño, se sigue que
	los puntos medios de todos los rectángulos corresponden a los puntos:

		$$ (\frac{1}{4}, \frac{7}{4}), (\frac{3}{4}, \frac{7}{4}),
			(\frac{5}{4}, \frac{7}{4}), (\frac{7}{4}, \frac{7}{4}) $$
		$$ (\frac{1}{4}, \frac{5}{4}), (\frac{3}{4}, \frac{5}{4}),
			(\frac{5}{4}, \frac{5}{4}), (\frac{7}{4}, \frac{5}{4}) $$
		$$ (\frac{1}{4}, \frac{3}{4}), (\frac{3}{4}, \frac{3}{4}),
			(\frac{5}{4}, \frac{3}{4}), (\frac{7}{4}, \frac{3}{4}) $$
		$$ (\frac{1}{4}, \frac{1}{4}), (\frac{3}{4}, \frac{1}{4}),
			(\frac{5}{4}, \frac{1}{4}), (\frac{7}{4}, \frac{1}{4}) $$

	así, evaluando los puntos bajo la función, se obtiene:

	$$ f(\frac{1}{4}, \frac{7}{4}) =  - \frac{13}{4}, \quad f(\frac{3}{4}, \frac{7}{4}) =  - \frac{11}{4} , \quad
	   f(\frac{5}{4}, \frac{7}{4}) =  - \frac{9}{4} , \quad f(\frac{7}{4}, \frac{7}{4}) = - \frac{7}{4} $$

	$$ f(\frac{1}{4}, \frac{5}{4}) =  - \frac{9}{4}, \quad f(\frac{3}{4}, \frac{5}{4}) =  - \frac{7}{4} , \quad
	   f(\frac{5}{4}, \frac{5}{4}) =  - \frac{5}{4} , \quad f(\frac{7}{4}, \frac{5}{4}) = - \frac{3}{4} $$

	$$ f(\frac{1}{4}, \frac{3}{4}) =  - \frac{5}{4}, \quad f(\frac{3}{4}, \frac{3}{4}) =  - \frac{3}{4} , \quad
	   f(\frac{5}{4}, \frac{3}{4}) =  - \frac{1}{4} , \quad f(\frac{7}{4}, \frac{3}{4}) = \frac{1}{4} $$

	$$ f(\frac{1}{4}, \frac{1}{4}) =  - \frac{1}{4}, \quad f(\frac{3}{4}, \frac{1}{4}) = \frac{1}{4} , \quad
	   f(\frac{5}{4}, \frac{1}{4}) =  \frac{3}{4} , \quad f(\frac{7}{4}, \frac{1}{4}) = \frac{5}{4} $$

	Se tiene, que como los valores de $x^{*}_{k}$ y $ y^{*}_{k}$ van cambiando
	cada $\frac{1}{2}$ (por estar seleccionados en el centro de los rectángulos),
	se tiene:
		$$ \Delta A_k = \Delta x^{*}_{k} \Delta y^{*}_{k}
					  = \frac{1}{2} \cdot \frac{1}{2} = \frac{1}{4}$$
	 luego, la suma de Riemann resultante está dada por:

	$$ \sum_{k=1}^{16} f(x^{*}_{k}, y^{*}_{k}) \Delta A_{k} = - \frac{1}{4}(1) +
	  	\frac{1}{4}(\frac{1}{4}) + \frac{3}{4}(\frac{1}{4}) + \frac{5}{4}(\frac{1}{4}) + (- \frac{5}{4})(\frac{1}{4}) +
		(- \frac{3}{4})(\frac{1}{4}) + (- \frac{1}{4})(\frac{1}{4}) + \frac{1}{4}(\frac{1}{4})  $$
	$$ + (- \frac{9}{4})(\frac{1}{4}) + (- \frac{7}{4})(\frac{1}{4}) + (- \frac{5}{4})(\frac{1}{4}) + (- \frac{3}{4})(\frac{1}{4})
	   + (- \frac{13}{4})(\frac{1}{4}) + (- \frac{11}{4})(\frac{1}{4}) + (- \frac{9}{4})(\frac{1}{4}) + (- \frac{7}{4})(\frac{1}{4}) $$
	$$  = \frac{1}{4} \Big[ - \frac{1}{4} + \frac{1}{4} + \frac{3}{4} + \frac{5}{4} + - \frac{5}{4}
		- \frac{3}{4} - \frac{1}{4} + \frac{1}{4}
	    - \frac{9}{4} - \frac{7}{4} - \frac{5}{4} - \frac{3}{4}
	    - \frac{13}{4} - \frac{11}{4} - \frac{9}{4} - \frac{7}{4}  \Big]$$

	$$  = \frac{1}{4} \Big[ - \frac{9}{4} - \frac{7}{4} - \frac{5}{4} - \frac{3}{4}
	      - \frac{13}{4} - \frac{11}{4} - \frac{9}{4} - \frac{7}{4} \Big] $$
	$$ = \frac{1}{4} \Big[ - \frac{64}{4} \Big] = \frac{1}{4}[ -16]  = -4 $$

	\textit{ (b) Compare the result in part (a) to the exact value of the integral. }\\

		Ahora, calculando la integral doble para la región delimitada por
		$$ 0 \leq x \leq 2, 0 \leq y \leq 2  $$

		para la función $ f(x,y) = x - 2y $ se tiene:

		$$ \iint\limits_{R} f(x,y)\mathrm{d}A = \int_{0}^{2} \int_{0}^{2} x - 2y \; dx dy  $$
		$$  = \int_{0}^{2} [ \frac{x^2}{2} -2yx] \Big|_0^2 dy $$
		$$  = \int_{0}^{2} \frac{4}{2} -2y(2) \; dy $$
		$$  = \int_{0}^{2} 2 - 4y \; dy $$
		$$  = [ 2y - \frac{4y^2}{2}] \Big|_0^2 $$
		$$  = [ 2y - 2y^2 ] \Big|_0^2 $$
		$$  = 4 - 8  = - 4 $$

		Así, la diferencia ente el resultado en la parte (a) y la parte (b) es nula\\

	\textit{\textbf{23 - 26 True - False.} Determine whether the statement is
	true or false. Explain your answer.\\}

	\textit{\textbf{23.} In the definition of a double integral}

		$$  \iint\limits_{R} f(x,y)\mathrm{d}A = \lim\limits_{n \rightarrow + \infty}
		 	\sum_{k=1}^{n} f(x^{*}_{k}, y^{*}_{k}) \Delta A_{k} $$

	\textit{ the symbol $ \Delta A_{k}$ represents a rectangular region within $R$
	 		from which the point $ (x^{*}_{k}, y^{*}_{k}) $ is taken.}\\

	Decimos que el enunciado es \textbf{cierto}. Considerememos el siguiente texto extraído
	del libro \textit{Calculus - 10th Edition por Anton /Bivens} para realizar el
	cálculo del volumen entre una superficie $ z = f(x,y)$ y una región $R$, donde
	las secciones resaltadas con negritas sustentan la respuesta: \\

	- Using lines parallel to the coordinate axes, divide the rectangle
	enclosing the region R into subrectangles, and exclude from consideration
	all those subrectangles that contain any points outside of R.
	\textbf{This leaves only rectangles that are subsets of R.} Assume that there are $n$
	such rectangles, and \textbf{denote the area of the kth such rectangle by $A_k$.} \\

	- Choose any arbitrary point in each subrectangle, \textbf{and denote the point in
	the kth subrectangle by $ (x^{*}_{k}, y^{*}_{k}) $.} The product
	$ f(x^{*}_{k}, y^{*}_{k}) \Delta A_k$  is the volume of a rectangular
	parallelepiped with base area $A_k$ and height $f(x^{*}_{k}, y^{*}_{k})$,so the sum

		$$ \sum_{k=1}^{n} f(x^{*}_{k}, y^{*}_{k}) \Delta A_{k} $$

	can be viewed as an approximation to the volume V of the entire solid.\\

	\textit{\textbf{24.} If R is the rectangle $ \{(x,y):1 \leq x \leq 4,0 \leq y \leq 3\} $
			and $\int_{0}^{3} f(x,y) dy = 2x$, then}

		$$ \iint\limits_{R} f(x,y)\mathrm{d}A = 15 $$
	\
	Se tiene que el teorema de Fubini enuncia:\\
	\textit{ Let R be the rectangle defined by the inequalities $a \leq x \leq b$, $c \leq y \leq d$ }.\\
	\textit{ If $f(x, y)$ is continuous on this rectangle, then }

		$$ \iint\limits_{R} f(x,y)\mathrm{d}A = \int_{c}^{d} \int_{a}^{b} f(x,y) dx dy
			= \int_{a}^{b} \int_{c}^{d} f(x,y) dy dx  $$

	Así, aplicando el teorema de Fubini, se sigue:
		$$ \iint\limits_{R} f(x,y)\mathrm{d}A = \int_{1}^{4} \int_{0}^{3} f(x,y) dy dx
		 	= \int_{1}^{4} 2x dx = \frac{2x^2}{2} \Big|_1^4 = x^2 \Big|_1^4 = 16 - 1 = 15 $$

		Así, el enunciado resulta ser \textbf{cierto}.\\

	\textit{\textbf{25.} If R is the rectangle $ \{(x,y):1 \leq x \leq 5,2 \leq y \leq 4\} $,
			then}

		$$ \iint\limits_{R} f(x,y)\mathrm{d}A = \int_{1}^{5} \int_{2}^{4} f(x,y) dx dy $$

	De igual forma, aplicando el teorema de Fubini, como en ele ejercicio anterior se tiene:

		$$ \iint\limits_{R} f(x,y)\mathrm{d}A = \int_{2}^{4} \int_{1}^{5} f(x,y) dx dy $$

			o bien:

		$$ \iint\limits_{R} f(x,y)\mathrm{d}A = \int_{1}^{5} \int_{2}^{4} f(x,y) dy dx $$

	pero la integral propuesta no puede suceder, dados los límites que caractrizan
	al rectángulo $R$ por lo que, se sigue que el enunciado es \textbf{falso}.\\

	\textit{\textbf{26.} Suppose that for some region R in the xy-plane}
			$$ \iint\limits_{R} f(x,y)\mathrm{d}A = 0 $$
	\textit{ If $R$ is subdivided into two regions $R_1$ and $R_2$, then }
			$$ \iint\limits_{R_1} f(x,y)\mathrm{d}A = - \iint\limits_{R_2} f(x,y)\mathrm{d}A  $$

	Citando al libro \textit{Calculus} de Anton-Bivens, en la página 1006, en un
	apartado de Propiedades de Integales Dobles, menciona lo siguiente:\\

	\textit{It is evident intuitively that if $f(x, y)$ is nonnegative on a
	region R, then subdividing R into two regions $R_1$ and $R_2$ has the effect
	of subdividing the solid between R and $z = f (x , y)$ into two solids,
	the sum of whose volumes is the volume of the entire solid. This suggests
	the following result, which holds even if $f$ has negative values:}

		$$ \iint\limits_{R} f(x,y)\mathrm{d}A
			= \iint\limits_{R_1} f(x,y)\mathrm{d}A + \iint\limits_{R_2} f(x,y)\mathrm{d}A  $$

	Ahora, aplicando dicha propiedad al problema, se sigue por hipótesis:

		$$ 0 = \iint\limits_{R} f(x,y)\mathrm{d}A
			= \iint\limits_{R_1} f(x,y)\mathrm{d}A + \iint\limits_{R_2} f(x,y)\mathrm{d}A  $$

	y así :

		$$ 0 = \iint\limits_{R_1} f(x,y)\mathrm{d}A + \iint\limits_{R_2} f(x,y)\mathrm{d}A  $$

	con lo que se sigue, conmutando la igualdad :

		$$ \iint\limits_{R_1} f(x,y)\mathrm{d}A + \iint\limits_{R_2} f(x,y)\mathrm{d}A = 0$$

	y de allí se obtiene :

		$$ \iint\limits_{R_1} f(x,y)\mathrm{d}A = - \iint\limits_{R_2} f(x,y)\mathrm{d}A $$

	Con lo cual, concluimos que el enunciado resulta ser \textbf{cierto}.\\

	\textit{\textbf{27} Demuestre que \[\int_{R}\int f(x,y)dA = \int_a^b[\int_c^dg(x)h(y)dy]dx\]}\\
	Por el teorema de Foubini tenemos que:
	\[\int_{R}\int f(x,y)dA = \int_a^b[\int_c^dg(x)h(y)dy]dx\]
	\[\int_a^b g(x)[\int_c^dh(y)dy ]dx\]
	\[[\int_a^b g(x) dx][\int_c^d h(y) dy]\]
	Por lo tanto la propiedad se cumple.\\
	\textit{\textbf{28.}Use the result in Exercise 27 to evaluate the integral
\[\int^{ln 3}_{0} \int_1 ^1 \sqrt{e^y + 1}\tan x dx dy\] by inspection. Explain your reasoning}\\

Del ejercicio $27$ tenemos que:
\[\int_{R}\int f(x,y)dA = \int_a^b[\int_c^dg(x)h(y)dy]dx\]
\[\int_a^b g(x)[\int_c^dh(y)dy ]dx\]
\[[\int_a^b g(x) dx][\int_c^d h(y) dy]\]
Entonces, sustituyendo tenemos que
\[\int^{ln 3}_{0} \int_1 ^1 \sqrt{e^y + 1}\tan x dx dy=[\int_o^{ln3} \sqrt{e^y + 1}dy][\int_1^1tan x dx]\]
Sabemos que $tan x$ es una función impar, por lo que al ser evaluada con $1$ y $-1$ es igual a cero, por lo tanto la integral es $0$\

\textit{\textbf{33.}Evaluate the integral by choosing a convenient order of integration:}
\[\int \int_{R}xcos(xy)\cos^2 \pi x dA; R = [0, \dfrac{1}{2}] x [0, \pi] \]

Tenemos la integral:
 \[\int_{0}^{1/2}\int_{0}^\pi x\cos(xy) \cos^2 (\pi x) dy dx\]
El orden por el que optaremos será el siguiente:
\[\int_{0}^{1/2} \left[ \int_{0}^{\pi} x\cos(xy) \cos^2(\pi x)dy \right] dx \]
Resolvemos primero:
\[\int_{0}^{\pi} x\cos(xy) \cos^2(\pi x)dy\]
Como estamos integrando respecto a $y$, entonces podemos sacar las constantes, así:
\[x\cos^2(\pi x)\int_{0}^{\pi} \cos(xy) dy\]
Hacemos cambio de variable, sea $u = xy$, entonces $du = x dy$ y $dy = du \left( \dfrac{1}{x} \right)$
\[x\cos^2(\pi x)\int_{0}^{\pi} \cos(xy) dy = x\cos^2(\pi x)\left[ \dfrac{1}{x} \int_{0}^{\pi}\cos u du \right] \]
\[= x\cos^2(\pi x)\left[\dfrac{1}{x}\left(\sin(xy)|_{o}^{\pi}\right) \right]\]
\[= x\cos^2(\pi x)\left[\dfrac{1}{x} \left(\sin(\pi x)- \sin(0) \right) \right]= x\cos^2(\pi x)\left[ \dfrac{1}{x} \left(\sin(\pi x)\right) \right]\]
\[= \cos^2(\pi x) \sin(\pi x) \left(\dfrac{1}{x} \right)= \cos^2(\pi x)\sin(\pi x)\]
Entonces :
\[\int_{0}^{1/2} \left[ \int_{0}^{\pi} x\cos(xy) \cos^2(\pi x)dy \right] dx = \int_{0}^{1/2}\cos^2(\pi x)\sin(\pi x) dx\]
\[=\int_{0}^{1/2}\sin(\pi x)\cos^2(\pi x) dx\]
Sea $u= \pi x$, entonces $du= \pi dx$ y $dx= \dfrac{1}{\pi}du$, así:
\[=\dfrac{1}{\pi} \int_{0}^{\pi/2}\sin u \cos^2u du\]
Hagamos un cambio de variable más, sea $v = \cos u$, entonces $dv = -\sin(u)du$ por lo que:
\[= \dfrac{1}{\pi} \int_{0}^{\pi / 2}v^2 dv = -\dfrac{1}{\pi}\left[\dfrac{v^3}{3}|_{0}^{\pi / 2}\right]\]
\[=-\dfrac{1}{\pi}\left[\dfrac{\cos(\pi x)}{3} |_{0}^{1/2}\right]= -\dfrac{1}{\pi}\left[\dfrac{\cos(\pi x)}{3}|_{0}^{1/2}\right]\]
\[-\dfrac{1}{\pi}\left[\dfrac{1}{3} \cos(\pi / 2)- \cos(0)\right]= \dfrac{1}{3\pi}\]

\textit{\textbf{37} }

	\textit{\textbf{39.} Suppose that the temperature in degrees Celsius at a
	point $(x,y)$ on a flat metal plate is $ T(x,y)=10 - 8x^2 - 2y^2$, where x
	and y are in meters. Find the average temperature of the rectangular portion
	of the plate for which $0 \leq x \leq 1$ and $0 \leq y \leq 2$.}\\

	Se tiene, pues que para calcular la temperatura promedio de un área
	determinada, esta está dada mediante la integral doble :

		$$ \frac{1}{A}  \iint\limits_{R} T(x,y)\mathrm{d}A   $$

	donde A representa al área $ 0 \leq x \leq 1$, $ 0 \leq y \leq 2 $ de la
	placa metálica.\\

=======

\end{itemize}
	\section{Anton-Bivens-Davis | Sección 14.1}
	\textit{\textbf{10.} $\int^{\pi}_{\pi / 2}\int^2_1 x \cos xy dy dx$}\
En nuestra herramienta de cálculo pusimos :
N[Integrate[x Cos[x y], {x, 1, 2}, {y, Pi/2, Pi}]] y nos arroja :
\[int^{\pi}_{\pi /2}\int^2_1 x \cos xy dy dx = - \dfrac{4}{\pi} \approx -1.27324\]

	\textit{\textbf{18.} (a) Let $f(x,y) = x - 2y$,and as shown in Exercise 17,
	let the rectangle $ R = [0, 2] x [0, 2]$ be subdivided into 16 subrectangles.\\
	\newline
	Take $(x^{*}_{k}, y^{*}_{k})$ to be the center of the $k-th$ rectangle, and approximate
	the double integral of f over R by the resulting Riemann sum.}\\

	Se tiene, que como $(x^{*}_{k}, y^{*}_{k})$ corresponden a los puntos medios
	de cada uno de los $k$ rectángulos, es necesario calcular el valor de cada
	uno de los puntos bajo la función $f$, así, como el rectángulo R está
	comprendido entre

		$$ 0 \leq x \leq 2, 0 \leq y \leq 2  $$

	y dicho rectángulo se divide en 16 rectángulos de igual tamaño, se sigue que
	los puntos medios de todos los rectángulos corresponden a los puntos:

		$$ (\frac{1}{4}, \frac{7}{4}), (\frac{3}{4}, \frac{7}{4}),
			(\frac{5}{4}, \frac{7}{4}), (\frac{7}{4}, \frac{7}{4}) $$
		$$ (\frac{1}{4}, \frac{5}{4}), (\frac{3}{4}, \frac{5}{4}),
			(\frac{5}{4}, \frac{5}{4}), (\frac{7}{4}, \frac{5}{4}) $$
		$$ (\frac{1}{4}, \frac{3}{4}), (\frac{3}{4}, \frac{3}{4}),
			(\frac{5}{4}, \frac{3}{4}), (\frac{7}{4}, \frac{3}{4}) $$
		$$ (\frac{1}{4}, \frac{1}{4}), (\frac{3}{4}, \frac{1}{4}),
			(\frac{5}{4}, \frac{1}{4}), (\frac{7}{4}, \frac{1}{4}) $$

	así, evaluando los puntos bajo la función, se obtiene:

	$$ f(\frac{1}{4}, \frac{7}{4}) =  - \frac{13}{4}, \quad f(\frac{3}{4}, \frac{7}{4}) =  - \frac{11}{4} , \quad
	   f(\frac{5}{4}, \frac{7}{4}) =  - \frac{9}{4} , \quad f(\frac{7}{4}, \frac{7}{4}) = - \frac{7}{4} $$

	$$ f(\frac{1}{4}, \frac{5}{4}) =  - \frac{9}{4}, \quad f(\frac{3}{4}, \frac{5}{4}) =  - \frac{7}{4} , \quad
	   f(\frac{5}{4}, \frac{5}{4}) =  - \frac{5}{4} , \quad f(\frac{7}{4}, \frac{5}{4}) = - \frac{3}{4} $$

	$$ f(\frac{1}{4}, \frac{3}{4}) =  - \frac{5}{4}, \quad f(\frac{3}{4}, \frac{3}{4}) =  - \frac{3}{4} , \quad
	   f(\frac{5}{4}, \frac{3}{4}) =  - \frac{1}{4} , \quad f(\frac{7}{4}, \frac{3}{4}) = \frac{1}{4} $$

	$$ f(\frac{1}{4}, \frac{1}{4}) =  - \frac{1}{4}, \quad f(\frac{3}{4}, \frac{1}{4}) = \frac{1}{4} , \quad
	   f(\frac{5}{4}, \frac{1}{4}) =  \frac{3}{4} , \quad f(\frac{7}{4}, \frac{1}{4}) = \frac{5}{4} $$

	Se tiene, que como los valores de $x^{*}_{k}$ y $ y^{*}_{k}$ van cambiando
	cada $\frac{1}{2}$ (por estar seleccionados en el centro de los rectángulos),
	se tiene:
		$$ \Delta A_k = \Delta x^{*}_{k} \Delta y^{*}_{k}
					  = \frac{1}{2} \cdot \frac{1}{2} = \frac{1}{4}$$
	 luego, la suma de Riemann resultante está dada por:

	$$ \sum_{k=1}^{16} f(x^{*}_{k}, y^{*}_{k}) \Delta A_{k} = - \frac{1}{4}(1) +
	  	\frac{1}{4}(\frac{1}{4}) + \frac{3}{4}(\frac{1}{4}) + \frac{5}{4}(\frac{1}{4}) + (- \frac{5}{4})(\frac{1}{4}) +
		(- \frac{3}{4})(\frac{1}{4}) + (- \frac{1}{4})(\frac{1}{4}) + \frac{1}{4}(\frac{1}{4})  $$
	$$ + (- \frac{9}{4})(\frac{1}{4}) + (- \frac{7}{4})(\frac{1}{4}) + (- \frac{5}{4})(\frac{1}{4}) + (- \frac{3}{4})(\frac{1}{4})
	   + (- \frac{13}{4})(\frac{1}{4}) + (- \frac{11}{4})(\frac{1}{4}) + (- \frac{9}{4})(\frac{1}{4}) + (- \frac{7}{4})(\frac{1}{4}) $$
	$$  = \frac{1}{4} \Big[ - \frac{1}{4} + \frac{1}{4} + \frac{3}{4} + \frac{5}{4} + - \frac{5}{4}
		- \frac{3}{4} - \frac{1}{4} + \frac{1}{4}
	    - \frac{9}{4} - \frac{7}{4} - \frac{5}{4} - \frac{3}{4}
	    - \frac{13}{4} - \frac{11}{4} - \frac{9}{4} - \frac{7}{4}  \Big]$$

	$$  = \frac{1}{4} \Big[ - \frac{9}{4} - \frac{7}{4} - \frac{5}{4} - \frac{3}{4}
	      - \frac{13}{4} - \frac{11}{4} - \frac{9}{4} - \frac{7}{4} \Big] $$
	$$ = \frac{1}{4} \Big[ - \frac{64}{4} \Big] = \frac{1}{4}[ -16]  = -4 $$

	\textit{ (b) Compare the result in part (a) to the exact value of the integral. }\\

		Ahora, calculando la integral doble para la región delimitada por
		$$ 0 \leq x \leq 2, 0 \leq y \leq 2  $$

		para la función $ f(x,y) = x - 2y $ se tiene:

		$$ \iint\limits_{R} f(x,y)\mathrm{d}A = \int_{0}^{2} \int_{0}^{2} x - 2y \; dx dy  $$
		$$  = \int_{0}^{2} [ \frac{x^2}{2} -2yx] \Big|_0^2 dy $$
		$$  = \int_{0}^{2} \frac{4}{2} -2y(2) \; dy $$
		$$  = \int_{0}^{2} 2 - 4y \; dy $$
		$$  = [ 2y - \frac{4y^2}{2}] \Big|_0^2 $$
		$$  = [ 2y - 2y^2 ] \Big|_0^2 $$
		$$  = 4 - 8  = - 4 $$

		Así, la diferencia ente el resultado en la parte (a) y la parte (b) es nula\\

	\textit{\textbf{23 - 26 True - False.} Determine whether the statement is
	true or false. Explain your answer.\\}

	\textit{\textbf{23.} In the definition of a double integral}

		$$  \iint\limits_{R} f(x,y)\mathrm{d}A = \lim\limits_{n \rightarrow + \infty}
		 	\sum_{k=1}^{n} f(x^{*}_{k}, y^{*}_{k}) \Delta A_{k} $$

	\textit{ the symbol $ \Delta A_{k}$ represents a rectangular region within $R$
	 		from which the point $ (x^{*}_{k}, y^{*}_{k}) $ is taken.}\\

	Decimos que el enunciado es \textbf{cierto}. Considerememos el siguiente texto extraído
	del libro \textit{Calculus - 10th Edition por Anton /Bivens} para realizar el
	cálculo del volumen entre una superficie $ z = f(x,y)$ y una región $R$, donde
	las secciones resaltadas con negritas sustentan la respuesta: \\

	- Using lines parallel to the coordinate axes, divide the rectangle
	enclosing the region R into subrectangles, and exclude from consideration
	all those subrectangles that contain any points outside of R.
	\textbf{This leaves only rectangles that are subsets of R.} Assume that there are $n$
	such rectangles, and \textbf{denote the area of the kth such rectangle by $A_k$.} \\

	- Choose any arbitrary point in each subrectangle, \textbf{and denote the point in
	the kth subrectangle by $ (x^{*}_{k}, y^{*}_{k}) $.} The product
	$ f(x^{*}_{k}, y^{*}_{k}) \Delta A_k$  is the volume of a rectangular
	parallelepiped with base area $A_k$ and height $f(x^{*}_{k}, y^{*}_{k})$,so the sum

		$$ \sum_{k=1}^{n} f(x^{*}_{k}, y^{*}_{k}) \Delta A_{k} $$

	can be viewed as an approximation to the volume V of the entire solid.\\

	\textit{\textbf{24.} If R is the rectangle $ \{(x,y):1 \leq x \leq 4,0 \leq y \leq 3\} $
			and $\int_{0}^{3} f(x,y) dy = 2x$, then}

		$$ \iint\limits_{R} f(x,y)\mathrm{d}A = 15 $$
	\
	Se tiene que el teorema de Fubini enuncia:\\
	\textit{ Let R be the rectangle defined by the inequalities $a \leq x \leq b$, $c \leq y \leq d$ }.\\
	\textit{ If $f(x, y)$ is continuous on this rectangle, then }

		$$ \iint\limits_{R} f(x,y)\mathrm{d}A = \int_{c}^{d} \int_{a}^{b} f(x,y) dx dy
			= \int_{a}^{b} \int_{c}^{d} f(x,y) dy dx  $$

	Así, aplicando el teorema de Fubini, se sigue:
		$$ \iint\limits_{R} f(x,y)\mathrm{d}A = \int_{1}^{4} \int_{0}^{3} f(x,y) dy dx
		 	= \int_{1}^{4} 2x dx = \frac{2x^2}{2} \Big|_1^4 = x^2 \Big|_1^4 = 16 - 1 = 15 $$

		Así, el enunciado resulta ser \textbf{cierto}.\\

	\textit{\textbf{25.} If R is the rectangle $ \{(x,y):1 \leq x \leq 5,2 \leq y \leq 4\} $,
			then}

		$$ \iint\limits_{R} f(x,y)\mathrm{d}A = \int_{1}^{5} \int_{2}^{4} f(x,y) dx dy $$

	De igual forma, aplicando el teorema de Fubini, como en ele ejercicio anterior se tiene:

		$$ \iint\limits_{R} f(x,y)\mathrm{d}A = \int_{2}^{4} \int_{1}^{5} f(x,y) dx dy $$

			o bien:

		$$ \iint\limits_{R} f(x,y)\mathrm{d}A = \int_{1}^{5} \int_{2}^{4} f(x,y) dy dx $$

	pero la integral propuesta no puede suceder, dados los límites que caractrizan
	al rectángulo $R$ por lo que, se sigue que el enunciado es \textbf{falso}.\\

	\textit{\textbf{26.} Suppose that for some region R in the xy-plane}
			$$ \iint\limits_{R} f(x,y)\mathrm{d}A = 0 $$
	\textit{ If $R$ is subdivided into two regions $R_1$ and $R_2$, then }
			$$ \iint\limits_{R_1} f(x,y)\mathrm{d}A = - \iint\limits_{R_2} f(x,y)\mathrm{d}A  $$

	Citando al libro \textit{Calculus} de Anton-Bivens, en la página 1006, en un
	apartado de Propiedades de Integales Dobles, menciona lo siguiente:\\

	\textit{It is evident intuitively that if $f(x, y)$ is nonnegative on a
	region R, then subdividing R into two regions $R_1$ and $R_2$ has the effect
	of subdividing the solid between R and $z = f (x , y)$ into two solids,
	the sum of whose volumes is the volume of the entire solid. This suggests
	the following result, which holds even if $f$ has negative values:}

		$$ \iint\limits_{R} f(x,y)\mathrm{d}A
			= \iint\limits_{R_1} f(x,y)\mathrm{d}A + \iint\limits_{R_2} f(x,y)\mathrm{d}A  $$

	Ahora, aplicando dicha propiedad al problema, se sigue por hipótesis:

		$$ 0 = \iint\limits_{R} f(x,y)\mathrm{d}A
			= \iint\limits_{R_1} f(x,y)\mathrm{d}A + \iint\limits_{R_2} f(x,y)\mathrm{d}A  $$

	y así :

		$$ 0 = \iint\limits_{R_1} f(x,y)\mathrm{d}A + \iint\limits_{R_2} f(x,y)\mathrm{d}A  $$

	con lo que se sigue, conmutando la igualdad :

		$$ \iint\limits_{R_1} f(x,y)\mathrm{d}A + \iint\limits_{R_2} f(x,y)\mathrm{d}A = 0$$

	y de allí se obtiene :

		$$ \iint\limits_{R_1} f(x,y)\mathrm{d}A = - \iint\limits_{R_2} f(x,y)\mathrm{d}A $$

	Con lo cual, concluimos que el enunciado resulta ser \textbf{cierto}.\\

	\textit{\textbf{28.}Use the result in Exercise 27 to evaluate the integral
	\[\int^{ln 3}_{0} \int^1_-1 \sqrt{e^y + 1}\tan x dx dy\]
	by inspection. Explain your reasoning}\ \\

	\textit{\textbf{33}}\\

	\textit{\textbf{39.} Suppose that the temperature in degrees Celsius at a
	point $(x,y)$ on a flat metal plate is $ T(x,y)=10 - 8x^2 - 2y^2$, where x
	and y are in meters. Find the average temperature of the rectangular portion
	of the plate for which $0 \leq x \leq 1$ and $0 \leq y \leq 2$.}\\

	Se tiene, pues que para calcular la temperatura promedio de un área
	determinada, esta está dada mediante la integral doble :

		$$ \frac{1}{A}  \iint\limits_{R} T(x,y)\mathrm{d}A   $$

	donde A representa al área $ 0 \leq x \leq 1$, $ 0 \leq y \leq 2 $ de la
	placa metálica.\\

	Luego, calculando la integral doble se tiene:

		$$ \iint\limits_{R} T(x,y)\mathrm{d}A  =
			\int_{0}^{2} \int_{0}^{1} 10 - 8x^2 - 2y^2 \mathrm{d}x \mathrm{d}y   $$

		$$ = \int_{0}^{2} [ 10x - \frac{8x^3}{3} - 2xy^2 ] \Big|_0^1 \mathrm{d}y $$
		$$ = \int_{0}^{2} 10 - \frac{8}{3} - 2y^2  \mathrm{d}y $$
		$$ = \int_{0}^{2} \frac{30}{3} - \frac{8}{3} - 2y^2 \mathrm{d}y $$
		$$ = \int_{0}^{2} \frac{22}{3} - 2y^2 \mathrm{d}y $$
		$$ = [ \frac{22}{3}y - \frac{2y^3}{3} ] \Big|_0^2$$
		$$ = \frac{44}{3} - \frac{16}{3} $$
		$$ = \frac{28}{3} $$

	Luego, se tiene que como el área de la placa metálica está comprendida entre
	$ 0 \leq x \leq 1 $ y $ 0 \leq y \leq 2 $, entonces se sigue:

	 	$$ A = (1 - 0) * (2 - 0) = 1 * 2 = 2  $$

	luego, sustituyendo en la fórmula para obtener la temperatura promedio, se
	tiene:
		$$ \frac{1}{A}  \iint\limits_{R} T(x,y)\mathrm{d}A  = \frac{1}{2} (\frac{28}{3}) $$

$$ = \frac{28}{6} = \frac{14}{3} $$

\textit{\textbf{43} Evalúe las integrales dadas. ¿Los resultados contradicen el teorema 14.1.3?}\\
(Evaluación en la hoja de cálculo).\\
No precisamente. Según el teorema 14.1.3 cambiar el orden de integración no altera el resultado final, pero al evaluar obtenemos volúmenes con signos distintos, que es lo que sucede al evaluar la función invirtiendo el orden de integración. Cómo vimos en clase, el volumen siempre va a ser el mismo.
\end{document}
